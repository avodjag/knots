%%% První kapitola

\chapter{Definice a vlastnosti Jonesova polynomu}




Definice, důkaz ekvivalence definic
Předpokládám reudenaistra a pod

Studium invariantů, polynomů
Skein relation
Uzel
Link
Pro linky? Uzly?
Diagramy - popsat?

Ukázat, že ampirichal knot má substituci (wiki říká, že přes kauffman)

Uzel, orientace, link.
Vhodné věci tučně

\section{začátek}

Při definování Jonesova polynomu je důležité rozlišovat mezi linkem a jeho diagramem, tedy vhodnému rovinnému nakreslení nějaké jeho projekce. Každý uzel má nekonečně mnoho diagramů, ovšem uzlový invariant každému přiřadí stejnou hodnotu (což je blbá věta)

V diagramu orientovaného linku rozlišujeme křížení s kladnou a zápornou orientací, viz obrázek.

Pro popis polynomů uzlech a lincích se používají skein (česky přadeno) vztahy.
Skein vztah popisuje, jaká je spojitost mezi polynomy tří linků $L_+$, $ L_-$ a $L_0$, jejichž diagramy jsou identické až na oblast jednoho křížení. V linku $L_+$ má toto křížení kladnou orientaci, v $L_-$ zápornou a v $L_0$ je křížení rozpojené, viz obrázek.

\section{Definice}

\begin{definice}\label{def01:1}
Jonesův polynom orientovaného linku $L$ je laurentův polynom značený v proměnné $\sqrt t$ (tj. polynom v $Z[\sqrt t, \sqrt{t^{-1}}]$), značený $V_K(t)$ , který
\begin{itemize}
\item
je invariantní vůči *** transformacím,
\item 
  je normalizovaný, tedy polynom  $V_\circlearrowleft$ triviálního uzlu má hodnotu $1$ ,
\item  
splňuje skein vztah 
$$ \frac{1}{t} V_{L+} - t V_{L-} = (\sqrt{t}  + \frac{1}{\sqrt{t}}) V_{L_0}$$
\end{itemize}
\end{definice}

\begin{lemma}
Lemma o tom, jak se spočítá polynom sjednocení unknots. Vzorec
\end{lemma}
\begin{dukaz}
Pro dvě kružnice ze skein, pro dvě to plyne indukcí.
\end{dukaz}

\begin{pozn}
Z každého diagramu uzlu lze změnou křížení z kladného na záporné či obraceně získat diagram triviálního uzlu. Z každého diagramu linku tedy můžeme změnou křížení získat diagram sjednocení triviálních uzlů, jejichž polynom známe podle předchozího lemmatu. Jonesův polynom každého linku lze tedy díky skein vztahu rekurzivně spočítat z jeho libovolného diagramu. Definice je tím pádem korektní.
\end{pozn}

Definice Jonesova polynomu pomocí skein vztahů není příliš vhodná pro algoritmický výpočet. K němu použijeme ekvivalentní definici založenou na použití tzv. závorkového polynomu (bracket polynomial, Kauffman bracket).

\section{Závorkový polynom}
Závorkový polynom je definovaný pouze pro diagramy neorientovaných linků (tedy nikoli pro samotné linky). Je počítán z jednodušších uzlů.

\begin{definice}\label{def01:2}
Závorkový polynom neorientovaného diagramu $D$, značení $\langle D \rangle$, je Laurentův polynom v proměnné $A$, definovaný třemi odvozovacími pravidly:
\begin{enumerate}[i.]
\item
$ \langle \bigcirc  \rangle = 1$, kde $\bigcirc$ značí diagram s jednou komponentou bez křížení
\item
$ \langle krizeni  \rangle = A  \langle vert \rangle + A^{-1}  \langle hor \rangle $, kde $krizeni$ značí diagram obsahující křížení, $vert$ je diagram, který je shodný až na dané křížení, které je vertikální rozpojeno a $hor$ je diagram, v němž je křížení rozpojeno horizontálně.
\item
$ \langle D \cup \bigcirc \rangle = (-A^2 - A^{-2}) \langle D \rangle$, kde $D \cup \bigcirc $ značí sjednocení diagramu $D$ a diagramu s jednou komponentou bez křížení.
\end{enumerate}

\end{definice}

\begin{dusl}
$ \langle krizeni opacne  \rangle = A  \langle hor \rangle + A^{-1}  \langle vert \rangle $
\end{dusl}

\begin{lemma}
$ \langle smycka \rangle = A^{-3} \langle odsmycka \rangle$
$ \langle smycka naopak \rangle = A^{3} \langle odsmycka \rangle$
\end{lemma}
\begin{dukaz}
Par obrazku
\end{dukaz}

Je známý výsledek, že dva diagramy znázorňují stejný lnk (jsou ekvivalentní), pokud mezi nimi existuje série Reidematreových pohybů. Předchozí lemma nám říká, že závorkový polynom není invariantní vůči prvnímu typu Reidemastra. Je ovšem invariantní vůči zbylým typům.

\begin{tvrz}
Závorkový polynom je invariantní vůči druhému Reidemastrovi
\end{tvrz}
\begin{dukaz}
Par obrazku
\end{dukaz}

\begin{dusl}
Závorkový polynom je invariantní vůči třetímu Reidemastrovi.
\end{dusl}
\begin{dukaz}
Par obrazku
\end{dukaz}

Potřebujeme tedy nějakou úpravu, aby to fungovalo. Měřit míru zakroucení, writhe.

\begin{definice}
Zakroucení (writhe) orientovaného diagramu $D$ je součet znamení všech křížení v $D$, značí se $w(D)$.
\end{definice}

\begin{definice}
Normalizovaný závorkový polynom orientovaného linku $L$ definijeme $X(L) = (-A^3)^{-w(D)}\langle D \rangle$, kde je libovolný diagram linku $L$.
\end{definice}

Korektnost definice plyne z následujícího tvrzení.

\begin{tvrz}
Normalizovaný závorkový polynom je uzlový (linkový) invariant.
\end{tvrz}
\begin{dukaz}
Již víme, že je invariatní vůčí dva a tři, podle lemmatu bla bla je i podle jedna a je to hotovo.
\end{dukaz}

\begin{veta}
Normalizovaný závorkový polynom se substituovanou proměnnou je roven Jonesovu polynomu.
\end{veta}
\begin{dukaz}
Jedna sedi podle tvrzení
Dva sedi podle definice
Tři se musí nějak dokázat
\end{dukaz}



Vlastnosti:
uzly mají jen celočíselné, amperichal knots, 

Zminit, jake polynomy jsou zobecněním Jonesova?

Měla bych také říct, že je otevřená otázka, jestli má nějaký ne unknot polynom jedna.