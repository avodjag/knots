%%% První kapitola

\chapter{Definice a vlastnosti Jonesova polynomu}

\section{Co uvnitř}



Definice, důkaz ekvivalence definic, \#P 
Předpokládám reudenaistra a pod

Studium invariantů, polynomů
Skein relation
Uzel
Link
Pro linky? Uzly?
Diagramy - popsat?

Ukázat, že ampirichal knot má substituci (wiki říká, že přes kauffman)

Uzel, orientace, link.

\section{začátek}
Popis kladných a záporných křížení

Pro popis polynomů na uzlech se používají skein (česky přadeno) vztahy.
Skein vztah popisuje, jaký je vztah mezi polynomy tří linků $L_+$, $ L_-$ a $L_0$, jejichž diagramy jsou identické až na oblast jednoho křížení. V linku $L_+$ má toto křížení kladnou orientaci, v $L_-$ zápornou a v $L_0$ je křížení rozpojené, viz obrázek.

\section{Definice}

\begin{definice}\label{def01:1}
Jonesův polynom orientovaného uzlu $K$ je laurentův polynom značený $V_K(t)$ v proměnné $\sqrt t$ (tj. polynom v $Z[\sqrt t, \sqrt{t^{-1}}]$), který
\begin{itemize}
\item
je uzlový invariant
\item 
  je normalizovaný, tedy polynom triviálního uzlu $V_\circlearrowleft$ má hodnotu $1$ 
\item  
splňuje skein vztah 
$$ \frac{1}{t} V_{K+} - t V_{K-} = (\sqrt{t}  + \frac{1}{\sqrt{t}}) V_{K_0}$$
\end{itemize}
\end{definice}

Zmena, musi to byt s linky

Korektnost definice plyne z faktu, že z každého uzlového diagramu lze změnou křížení z kladného na záporné či obracéně získat diagram triviálního uzlu. Jonesův polynom každého uzlu lze tedy díky skein vztahu rekurzivně spočítatz jeho libovolného diagramu.

Ekvivalentní definice Jonesova polynomu, kterou použijeme v k výpočtu, je založena na závorkovém polynomu (bracket polynomial, Kauffman bracket).

\section{Závorkový polynom}
Závorkový polynom je definovaný pouze pro diagramy neorientovaných linků (tedy nikoli pro samotné linky). Je počítán z jednodušších uzlů.

\begin{definice}\label{def01:2}
Závorkový polynom neorientovaného diagramu $D$, značení $\langle D \rangle$, je Laurentův polynom v proměnné $A$, definovaný třemi odvozovacími pravidly:
\begin{enumerate}[i.]
\item
$ \langle \bigcirc  \rangle = 1$, kde $\bigcirc$ značí diagram s jednou komponentou bez křížení
\item
$ \langle krizeni  \rangle = A  \langle vert \rangle + A^{-1}  \langle hor \rangle $, kde $krizeni$ značí diagram obsahující křížení, $vert$ je diagram, který je shodný až na dané křížení, které je vertikální rozpojeno a $hor$ je diagram, v němž je křížení rozpojeno horizontálně.
\item
$ \langle D \cup \bigcirc \rangle = (-A^2 - A^{-2}) \langle D \rangle$, kde $D \cup \bigcirc $ značí sjednocení diagramu $D$ a diagramu s jednou komponentou bez křížení.
\end{enumerate}

\end{definice}

\begin{dusl}
$ \langle krizeni opacne  \rangle = A  \langle hor \rangle + A^{-1}  \langle vert \rangle $
\end{dusl}

\begin{lemma}
$ \langle smycka \rangle = A^{-3} \langle odsmycka \rangle$
$ \langle smycka naopak \rangle = A^{3} \langle odsmycka \rangle$
\end{lemma}
\begin{dukaz}
Par obrazku
\end{dukaz}

Dva diagramy jsou ekvivalentni, pokud mezi nimi existuje série Reidematre

Obrázek Reidemastra

\begin{tvrz}
Závorkový polynom je invariantní vůči druhému Reidemastrovi
\end{tvrz}
\begin{dukaz}
Par obrazku
\end{dukaz}

\begin{dusl}
Závorkový polynom je invariantní vůči třetímu Reidemastrovi.
\end{dusl}
\begin{dukaz}
Par obrazku
\end{dukaz}

\begin{definice}
Zakroucení (writhe) orientovaného diagramu $D$ je součet znamení všech křížení v $D$, značí se $w(D)$.
\end{definice}

\begin{definice}
Normalizovaný závorkový polynom orientovaného linku $L$ definijeme $X(L) = (-A^3)^{-w(D)}\langle D \rangle$, kde je libovolný diagram linku $L$.
\end{definice}

Korektnost definice plyne z následujícího tvrzení.

\begin{tvrz}
Normalizovaný závorkový polynom je uzlový (linkový) invariant.
\end{tvrz}
\begin{dukaz}
Již víme, že je invariatní vůčí dva a tři, podle lemmatu bla bla je i podle jedna a je to hotovo.
\end{dukaz}

\begin{veta}
Normalizovaný závorkový polynom se substituovanou proměnnou je roven Jonesovu polynomu.
\end{veta}
\begin{dukaz}
Jedna sedi podle tvrzení
Dva sedi podle definice
Tři se musí nějak dokázat
\end{dukaz}




