\documentclass[10pt]{article}

%% Přepneme na českou sazbu a fonty Latin Modern
\usepackage[czech]{babel}
\usepackage{lmodern}
\usepackage[T1]{fontenc}
\usepackage{textcomp}
\usepackage{subcaption}


%% Použité kódování znaků: obvykle latin2, cp1250 nebo utf8:
\usepackage[utf8]{inputenc}
\usepackage{pgf,tikz,pgfplots}
\pgfplotsset{compat=1.15}
\usepackage{mathrsfs}
\usetikzlibrary{arrows}
\usetikzlibrary{
  knots,
  hobby,
  decorations.pathreplacing,
  shapes.geometric,
  calc,
  decorations.markings,
  bending
}

\pagestyle{empty}
\begin{document}

\begin{tikzpicture}
\begin{knot}[
%  draft mode=crossings,
  flip crossing/.list={3,4}
]
\strand (1,0) circle[radius=2cm];
\strand[blue] (-1,0) circle[radius=2cm];
\strand[green] (0,{sqrt(3)}) circle[radius=2cm];
\strand[green] (0,-3) circle[radius=2cm];
\strand[green] (0,-3) circle[radius=2cm];
\node at (0,0) {A}
\end{knot}
\end{tikzpicture}



Druhy reid


\begin{tikzpicture}[scale = 0.5]
\draw (-1,-1) arc (270:90:1);
%\draw (-1, -1) -- (1, -1);
\draw (-1, -1) -- (-0.25, -1);
\draw (0.25, -1) -- (1, -1);
%\draw (-1, 1) -- (1, 1);
\draw (0, -1.3) -- (0, 1.3);

\draw (-1, 1) -- (-0.25, 1);
\draw (0.25, 1) -- (1, 1);
\end{tikzpicture}




\begin{tikzpicture}
\draw[fill] (1,1) circle [radius=0.05];
\node [below] at (1,1) {$P_1$};

\draw[fill] (2,-1/2) circle [radius=0.05];
\node [below] at (2,-1/2) {$P_2$};

\draw[fill] (2,3) circle [radius=0.05];
\node [above] at (2,3) {$P_3$};

\draw[fill] (-1,3/2) circle [radius=0.05];
\node [above] at (-1,3/2) {$P_4$};

\draw[fill] (-3,3/2) circle [radius=0.05];
\node [above] at (-3,3/2) {$P_5$};

\draw[fill] (-2,-1) circle [radius=0.05];
\node [below] at (-2,-1) {$P_6$};

\draw (-3,3/2) -- (2,3);
\draw (-3,3/2) -- (-1,3/2);
\draw (-3,3/2) -- (-2,-1);
\draw (-1,3/2)  -- (2,3) ;
\draw (-1,3/2)  -- (-2,-1) ;
\draw (-1,3/2)  -- (1,1) ;
\draw (-1,3/2)  -- (1,1) ;
\draw (2,3)   -- (1,1) ;
\draw (-2,-1) -- (1,1) ;
\draw (-2,-1) -- (2,-1/2);
\draw  (1,1)  -- (2,-1/2);
\draw  (2,3)-- (2,-1/2);


\end{tikzpicture}



\begin{tikzpicture}
\draw (0,0) arc (270:90:1);
\draw (0,0) -- (2,0);
\draw (0,2) -- (2,2);
\end{tikzpicture}

\begin{tikzpicture}
\clip(-2,-2) rectangle (7,1.3);
\draw (-1,1) .. controls (-2,2) and (-2,-2) .. (-1, -1);
\draw (-1,1) -- (1, -1);
\draw (-1,-1) -- (-0.25, -0.25);
\draw (0.25,0.25) -- (1,1);

\draw [thick, <->] (2,0) -- (3,0);

\draw (5,-1) arc (270:90:1);
\draw (5,-1) -- (7,-1);
\draw (5,1) -- (7,1);

\end{tikzpicture}

\begin{tikzpicture}
\clip(-2,-2) rectangle (7,1.3);

\draw (-1,-1) arc (270:90:1);
\draw (-1, -1) -- (1, -1);
\draw (-1, 1) -- (1, 1);

\draw [thick, <->] (2,0) -- (3,0);
\draw (5,-1) arc (270:90:1);
\draw (5,-1) -- (7,-1);
\draw (5,1) -- (7,1);

\end{tikzpicture}


\begin{tikzpicture}
\clip(-2,-2) rectangle (7,1.3);

\draw (-1,-1) arc (270:90:1);
%\draw (-1, -1) -- (1, -1);
\draw (-1, -1) -- (-0.25, -1);
\draw (0.25, -1) -- (1, -1);
%\draw (-1, 1) -- (1, 1);
\draw (0, -1.3) -- (0, 1.3);
\draw (4, -1.3) -- (4, 1.3);

\draw (-1, 1) -- (-0.25, 1);
\draw (0.25, 1) -- (1, 1);

\draw [thick, <->] (2,0) -- (3,0);
\draw (5.5,-1) arc (270:90:1);
\draw (5.5,-1) -- (7,-1);
\draw (5.5,1) -- (7,1);

\end{tikzpicture}


\begin{tikzpicture}
\clip(-4.5,-1.65) rectangle (4.5, 1.65);

\draw [thick, <->] (-1/2,0) -- (1/2,0);

\draw (-4,1) -- (-3.58, 0.72);
\draw (-3.4, 0.6) -- (-1,-1);
\draw (-4, -1) -- (-3.58, -0.72);
\draw (-3.4, -0.6) -- (-2.8, -0.2);
\draw (-2.2, 0.2) -- (-1, 1);

\draw (1,-1) -- (2.2, -0.2);
\draw (2.8, 0.2) -- (3.4, 0.6);
\draw (3.58, 0.72) -- (4, 1);
\draw (1,1) --  (3.4, -0.6) ;
\draw (3.58, -0.72) -- (4, -1);

\draw (-3.5, -1) -- (-3.5, 1);
\draw (3.5,-1) -- (3.5, 1);

\end{tikzpicture}



\begin{figure}[h]
\centering 
\begin{subfigure}{1\linewidth}\centering
\begin{tikzpicture}
\clip(-2,-2) rectangle (7,1.3);
\draw (-1,1) .. controls (-2,2) and (-2,-2) .. (-1, -1);
\draw (-1,1) -- (1, -1);
\draw (-1,-1) -- (-0.25, -0.25);
\draw (0.25,0.25) -- (1,1);

\draw [thick, <->] (2,0) -- (3,0);

\draw (5,-1) arc (270:90:1);
\draw (5,-1) -- (7,-1);
\draw (5,1) -- (7,1);

\end{tikzpicture}
\caption{Typ I} 
\end{subfigure}
\begin{subfigure}{1\linewidth}\centering
\begin{tikzpicture}
\clip(-2,-2) rectangle (7,1.3);

\draw (-1,-1) arc (270:90:1);
%\draw (-1, -1) -- (1, -1);
\draw (-1, -1) -- (-0.25, -1);
\draw (0.25, -1) -- (1, -1);
%\draw (-1, 1) -- (1, 1);
\draw (0, -1.3) -- (0, 1.3);
\draw (4, -1.3) -- (4, 1.3);

\draw (-1, 1) -- (-0.25, 1);
\draw (0.25, 1) -- (1, 1);

\draw [thick, <->] (2,0) -- (3,0);
\draw (5.5,-1) arc (270:90:1);
\draw (5.5,-1) -- (7,-1);
\draw (5.5,1) -- (7,1);

\end{tikzpicture} 
\caption{Typ II}
\end{subfigure}
\begin{subfigure}{1\linewidth}\centering
\begin{tikzpicture}
\clip(-4.5,-1.65) rectangle (4.5, 1.65);

\draw [thick, <->] (-1/2,0) -- (1/2,0);

\draw (-4,1) -- (-3.58, 0.72);
\draw (-3.4, 0.6) -- (-1,-1);
\draw (-4, -1) -- (-3.58, -0.72);
\draw (-3.4, -0.6) -- (-2.8, -0.2);
\draw (-2.2, 0.2) -- (-1, 1);

\draw (1,-1) -- (2.2, -0.2);
\draw (2.8, 0.2) -- (3.4, 0.6);
\draw (3.58, 0.72) -- (4, 1);
\draw (1,1) --  (3.4, -0.6) ;
\draw (3.58, -0.72) -- (4, -1);

\draw (-3.5, -1) -- (-3.5, 1);
\draw (3.5,-1) -- (3.5, 1);

\end{tikzpicture}
\caption{Typ III}
\end{subfigure}
\caption{Reidemeisterovy pohyby}
\end{figure}






\end{document}