%%% Tento soubor obsahuje definice různých užitečných maker a prostředí %%%
%%% Další makra připisujte sem, ať nepřekáží v ostatních souborech.     %%%

%%% Drobné úpravy stylu

% Tato makra přesvědčují mírně ošklivým trikem LaTeX, aby hlavičky kapitol
% sázel příčetněji a nevynechával nad nimi spoustu místa. Směle ignorujte.
\makeatletter
\def\@makechapterhead#1{
  {\parindent \z@ \raggedright \normalfont
   \Huge\bfseries \thechapter. #1
   \par\nobreak
   \vskip 20\p@
}}
\def\@makeschapterhead#1{
  {\parindent \z@ \raggedright \normalfont
   \Huge\bfseries #1
   \par\nobreak
   \vskip 20\p@
}}
\makeatother

% Toto makro definuje kapitolu, která není očíslovaná, ale je uvedena v obsahu.
\def\chapwithtoc#1{
\chapter*{#1}
\addcontentsline{toc}{chapter}{#1}
}

% Trochu volnější nastavení dělení slov, než je default.
\lefthyphenmin=2
\righthyphenmin=2

% Zapne černé "slimáky" na koncích řádků, které přetekly, abychom si
% jich lépe všimli.
\overfullrule=1mm

%%% Makra pro definice, věty, tvrzení, příklady, ... (vyžaduje baliček amsthm)

\theoremstyle{plain}
\newtheorem{veta}{Věta}
\newtheorem{lemma}[veta]{Lemma}
\newtheorem{tvrz}[veta]{Tvrzení}

\theoremstyle{plain}
\newtheorem{definice}{Definice}

\theoremstyle{remark}
\newtheorem*{dusl}{Důsledek}
\newtheorem*{pozn}{Poznámka}
\newtheorem*{prikl}{Příklad}

%%% Prostředí pro důkazy

\newenvironment{dukaz}{
  \par\medskip\noindent
  \textit{Důkaz}.
}{
\newline
\rightline{$\square$}  % nebo \SquareCastShadowBottomRight z balíčku bbding
}

%%% Prostředí pro sazbu kódu, případně vstupu/výstupu počítačových
%%% programů. (Vyžaduje balíček fancyvrb -- fancy verbatim.)

\DefineVerbatimEnvironment{code}{Verbatim}{fontsize=\small, frame=single}

%%% Prostor reálných, resp. přirozených čísel
\newcommand{\R}{\mathbb{R}}
\newcommand{\N}{\mathbb{N}}
% ja
\newcommand{\Z}{\mathbb{Z}}

%%% Užitečné operátory pro statistiku a pravděpodobnost
\DeclareMathOperator{\pr}{\textsf{P}}
\DeclareMathOperator{\E}{\textsf{E}\,}
\DeclareMathOperator{\var}{\textrm{var}}
\DeclareMathOperator{\sd}{\textrm{sd}}

%%% Příkaz pro transpozici vektoru/matice
\newcommand{\T}[1]{#1^\top}

%%% Vychytávky pro matematiku
\newcommand{\goto}{\rightarrow}
\newcommand{\gotop}{\stackrel{P}{\longrightarrow}}
\newcommand{\maon}[1]{o(n^{#1})}
\newcommand{\abs}[1]{\left|{#1}\right|}
\newcommand{\dint}{\int_0^\tau\!\!\int_0^\tau}
\newcommand{\isqr}[1]{\frac{1}{\sqrt{#1}}}

%%% Vychytávky pro tabulky
\newcommand{\pulrad}[1]{\raisebox{1.5ex}[0pt]{#1}}
\newcommand{\mc}[1]{\multicolumn{1}{c}{#1}}

\newcommand{\minuskriz}{
\begin{tikzpicture}[scale=0.4]
\draw (0,0) -- (0.33,0.33);
\draw  (0.66,0.66)-- (1,1);
\draw (0,1)  -- (1,0);
\end{tikzpicture}
}

\newcommand{\pluskriz}{\begin{tikzpicture}[scale=0.4]
\draw (0,0) -- (1,1);
\draw (0,1) -- (0.33, 0.66);
\draw (0.66, 0.33) -- (1,0);
\end{tikzpicture}}

\newcommand{\vertkriz}{\begin{tikzpicture} [scale=0.4, baseline]
\draw (0,0) .. controls (1/2,1/2)  .. (0,1);
\draw (1,0) .. controls (1/2,1/2)  .. (1,1);
\end{tikzpicture}}

\newcommand{\horkriz}{\begin{tikzpicture} [scale=0.4]
\draw (0,0) .. controls (1/2,1/2)  .. (1,0);
\draw (0,1) .. controls (1/2,1/2)  .. (1,1);
\end{tikzpicture}}

\newcommand{\plussmycka}{
\begin{tikzpicture}[baseline =\dimexpr-\fontdimen22\textfont2, scale = 0.25]
\begin{knot}[
consider self intersections=true,
clip width = 4,
end tolerance=1pt,
] 
\clip (0,-1) rectangle (2.5,1);
\strand (0,-1) .. controls (3,3) and (3,-3) ..  (0,1);
\end{knot}
\end{tikzpicture}}

\newcommand{\odsmycka}{
\begin{tikzpicture}[baseline =\dimexpr-\fontdimen22\textfont2,scale = 0.25]
\clip (0,-1) rectangle (2,1);
\draw (0,1) .. controls (2.5,0) ..  (0,-1);
\end{tikzpicture}}

\newcommand{\minussmycka}{
\begin{tikzpicture}[baseline =\dimexpr-\fontdimen22\textfont2,scale = 0.25]
\begin{knot}[
consider self intersections=true,
clip width = 4,
end tolerance=1pt,
] 
\clip (0,-1) rectangle (2.5,1);
\strand (0,1) .. controls (3,-3) and (3,3) ..  (0,-1);
\end{knot}
\end{tikzpicture}}

\newcommand{\AO}{
\begin{tikzpicture}[scale =0.3, baseline]
\draw (0,-1) arc (270:90:1);
\draw (0, 1) .. controls (1,1) .. (1, 1.5); 
\draw (1, -1.2) .. controls (1,1) .. (2, 1); 
\draw (0, -1) -- (0.75, -1);
\draw (1.25, -1) -- (2, -1);

\end{tikzpicture}}


\newcommand{\BO}{
\begin{tikzpicture}[scale =0.3, baseline]
\draw (0,-1) arc (270:90:1);
\draw (0, 1) .. controls (1,1) .. (1, 1.5); 
\draw (1, -1.2) .. controls (1,1) .. (2, 1); 
\draw (0, -1) -- (0.75, -1);
\draw (1.25, -1) -- (2, -1);

\end{tikzpicture}}


\newcommand{\CO}{
\begin{tikzpicture}[scale =0.3, baseline]
\draw (4, -1.3) -- (4, 1.3);


\draw (5.5,-1) arc (270:90:1);
\draw (5.5,-1) -- (7,-1);
\draw (5.5,1) -- (7,1);

\end{tikzpicture}}


\newcommand{\DO}{
\begin{tikzpicture}[scale =0.3, baseline]

\draw (1, 1.5) .. controls (1,0.5) .. (3, 0.5);
\draw (1, -1.5) .. controls (1,-0.5) .. (3, -0.5);


\end{tikzpicture}}


\newcommand{\EO}{
\begin{tikzpicture}[scale =0.3, baseline]
\draw (0,0) circle (0.4cm);

\draw (1, 1.5) .. controls (1,0.5) .. (3, 0.5);
\draw (1, -1.5) .. controls (1,-0.5) .. (3, -0.5);
\end{tikzpicture}}


\newcommand{\FO}{
\begin{tikzpicture}[scale =0.3, baseline]
\clip (-0.75, -1.5) rectangle (3, 1.5);

\draw (1, 1.5) .. controls (1,0.5) .. (3, 0.5);

\draw (0.75, -0.5) .. controls (-2, -0.5) and (1, 2) .. (1, -1.5);
\draw (1.25, -0.5) -- (3, -0.5);
\end{tikzpicture}}

\newcommand{\reiddva}{
\begin{tikzpicture}[scale =0.3, baseline]

\draw (-1,-1) arc (270:90:1);
%\draw (-1, -1) -- (1, -1);
\draw (-1, -1) -- (-0.25, -1);
\draw (0.25, -1) -- (1, -1);
%\draw (-1, 1) -- (1, 1);
\draw (0, -1.3) -- (0, 1.3);

\draw (-1, 1) -- (-0.25, 1);
\draw (0.25, 1) -- (1, 1);
\end{tikzpicture}}

\newcommand{\GO}{
\begin{tikzpicture}[scale =0.3, baseline]
\draw (1,-1) -- (2.2, -0.2);
\draw (2.8, 0.2) -- (3.4, 0.6);
\draw (3.58, 0.72) -- (4, 1);
\draw (1,1) --  (3.4, -0.6) ;
\draw (3.58, -0.72) -- (4, -1);

\draw (3.5,-1) -- (3.5, 1);
\end{tikzpicture}}

\newcommand{\HO}{
\begin{tikzpicture}[scale =0.3, baseline]
\draw (-4,1) -- (-3.58, 0.72);
\draw (-3.4, 0.6) -- (-1,-1);
\draw (-4, -1) -- (-3.58, -0.72);
\draw (-3.4, -0.6) -- (-2.8, -0.2);
\draw (-2.2, 0.2) -- (-1, 1);
\draw (-3.5, -1) -- (-3.5, 1);
\end{tikzpicture}}

\newcommand{\IO}{
\begin{tikzpicture}[scale =0.3, baseline]

\draw (-1,-1) arc (270:90:1);
%\draw (-1, -1) -- (1, -1);
\draw (-1, -1) -- (-0.25, -1);
\draw (0.25, -1) -- (1, -1);
%\draw (-1, 1) -- (1, 1);
\draw (0, -1.3) -- (0, 1.3);

\draw (-1, 1) -- (-0.25, 1);
\draw (0.25, 1) -- (1, 1);

\draw (-2.25, -1) -- (-2.25, 1);
\end{tikzpicture}}

\newcommand{\JO}{
\begin{tikzpicture}[scale =0.3, baseline]

\draw (3.4,-1) -- (3.4, 1);
\draw (4, 0.8) -- (3.7,0.8);
\draw (3.1,0.8) -- (1.5, 0.8);
\draw (4, -0.8) -- (3.7,-0.8);
\draw (3.1,-0.8) -- (1.5, -0.8);
\end{tikzpicture}}

\newcommand{\KO}{
\begin{tikzpicture}[scale =0.3, baseline]

\draw (-1,-1) arc (270:90:1);
\draw (-1, -1) -- (1, -1);
\draw (-1, 1) -- (1, 1);

\draw (-1, 1) -- (-0.25, 1);
\draw (0.25, 1) -- (1, 1);

\draw (-2.25, -1) -- (-2.25, 1);
\draw (-2.5, -1) -- (-2.5, 1);

\end{tikzpicture}}

\newcommand{\LO}{
\begin{tikzpicture}[scale =0.3, baseline]

%\draw (1,-1) arc (0:90:1);
\draw (1,-1) arc (-90:90:1);
\draw (1, -1) -- (0.25, -1);
\draw (-0.25, -1) -- (-1, -1);
\draw (-0, -1.3) -- (-0, 1.3);

\draw (1, 1) -- (0.25, 1);
\draw (-0.25, 1) -- (-1, 1);

\draw (2.25, -1) -- (2.25, 1);
\end{tikzpicture}}