\chapter*{Závěr}
\addcontentsline{toc}{chapter}{Závěr}

Hlavním cílem práce bylo odvodit a implementovat algoritmus na výpočet Jonesova polynomu. V první kapitole jsme dokázali souvislost mezi Jonesovým a~závorkovým polynomem.  Na základě tohoto vztahu jsme v druhé kapitole odvodili algoritmus založený na rekurzivním rozpojování křížení a postupném rozpojování zadaného diagramu. Dokázali jsme, že vhodnou volbou křížení, které rozpojíme, získáme algoritmus s časovou složitostí $\mathcal{O}\left(2^{0,832n}\right) $, kde $n$ značí počet křížení vstupního diagramu. Zároveň jsme nalezli typ diagramu, na němž algoritmus běží v čase $\Omega \left(2^{0,75n}\right) $, a získali jsme tak dolní odhad složitosti algoritmu.

V poslední části práce jsme shrnuli výsledky testování algoritmu na datech. Algoritmus jsme otestovali na tabulkových uzlech, torusových uzlech a různých větších náhodných lincích. Náhodné linky a uzly jsme vygenerovali pomocí převodu z rovinného grafu. U nejrychlejší varianty algoritmu  jsme na náhodných uzlech odhadli průměrnou časovou složitost výpočtu $\mathcal{O}\left(2^{0,468n+ o(n)}\right) $. Taktéž jsme zjistili, že na jednoduchém typu torusových uzlů běží algoritmus v kvadratickém čase. Na složitějších torusových uzlech dosahuje algoritmus horších výsledků, průměrnou složitost výpočtu Jonesova polynomu torusových uzlů jsme odhadli $\mathcal{O}\left(2^{0,533n+ o(n)}\right) $.
