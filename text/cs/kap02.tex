%%% Druhá kapitola
\chapter{Výpočet Jonesova polynomu}


Je to v tride number P  
Popis algoritmu, vypocet horniho odhadu, dolní odhad pro nějakou třídu uzlů, na které se to rozbije, skripta z počítačové algebry, důkaz správnosti algoritmu
Odhad složitosti?


Bylo by zajímavé identifikovat, pro které typy uzlů je algoritmus efektivní a pro které naopak dosahuje nejhorších výsledků.
Rychle na kanonickych nakreslenich torus uzlu a preclikovych uzlu.

Jak se používá velké O?
Už jsem viděla, že můžu použít O(0.83). Ale co s tím spodním?
Co problém, co algoritmus.

\section{Výpočetní složitost problému}

Podle (On the computational complexity of the Jones and Tutte polynomials) patří problém určení Jonesova polynomu alternujího uzlu do třídy složitosti \#P, dokonce je tento problém \# P-těžký. 

Třída \#P obsahuje problémy, jejichž cílem je určit počet přijímacích cest nedeterministického Turingova stroje, jedná se tedy o rozšíření problémů třídy NP. Například problém \# SAT znamená nejen určit, jestli existuje pravdivostní ohodnocení Boolovské formule, ale i spočítat, kolik takových ohodnocení existuje celkem.

Říct, co z toho plyne. Jako že nemůžu najít lineární algoritmus. Nebo bych jako byla fakt dobrá, kdyby ano.

\section{Algoritmus}
Nejdriv chcu popsat, že to tak jde. Pak dodat pseudokód.

Do jakých detailů? Jak popsat implementaci? Python? Důkaz správnosti - vždy se zastaví. Vstup s počtem křížení. Vstup je PD notace (to až nějak v implementaci).
Podkapitola na implementační detaily a vychytávky?

--------------

Náš algoritmus dostane na vstupu diagram orientovaného linku s $n$ kříženími zapsaný v PD notaci.

\subsection{PD notace} 

PD notace je zápis sestávajací ze čtveřice čísel pro každé křížení a jednoznačně popisuje daný diagram. Zápis diagramu v PD notaci se získá následovně: úseky mezi kříženími se očíslují po směru orientace linku čísly od $1$ do $2 n$. Každé křížení se označí čtyřmi přilehlými úseky, přičemž se začne úsekem, který do křížení vstupuje spodem, a pokračuje se s úseky navazujícími proti směru hodinových ručiček. Viz obrázek.

Linky

\subsection{Výpočet Jonesova polynomu ze závorkového polynomu}

Jak již bylo řečeno, k výpočtu Jonesova polynomu používáme závorkový polynom. Podle věty se Jonesův polynom získá z normalizovaného závorkového polynomu substitucí proměnné. 

\begin{algorithm}[H]
\DontPrintSemicolon
\KwData{Diagram linku $L$  s $n$ krížení}
\KwResult{Jonesův polynom v proměnné $t$}
\SetKwData{bracketPoly}{závorkovýPolynom}
\SetKwData{writhe}{zamotání}
\SetKwData{jones}{jonesůvPolynom}
\SetKwData{normal}{normalizovanýPolynom}
\SetKwFunction{Writhe}{Writhe}
\SetKwFunction{Bracket}{Bracket}
\SetKwFunction{Substituce}{Substituce}


\BlankLine

\bracketPoly $\leftarrow$ \Bracket{$L$}
\tcc*[r]{v proměnné $A$}

\writhe $\leftarrow$ \Writhe{$L$}

\normal $\leftarrow$ $ (-A^3)^{\writhe} \times \bracketPoly $
\jones $\leftarrow$ \Substituce{\normal , $A$, $t^{1/2}$}

\KwRet \jones 

\caption{Jonesův polynom} 
\end{algorithm}

V PD notaci lze jednoduše určit, jestli je křížení kladné, či záporné orientace, tedy zamotání spočítáme v $\mathcal{O} (n)$ čase.
spočítáme v lineárním čase vzhledem k počtu křížení.

Dále se budeme zabývat výpočtem závorkový polynom.



\subsection{Přímočarý výpočet závorkového polynomu}

Možná to celé nahradit diagramy.

Z definice závorkového polynomu plyne jednoduchý rekurzivní algoritmus.

Budu tomu říkat rozpojení křížení. Synové s $n-1$.

Háčky jsou možné, ale nesmí být jednoslovné názvy.
Polynomy rovnou psát v jake jsou proměnné.

\begin{algorithm}[H]

\DontPrintSemicolon

\SetKwData{cross}{krizeni}
\SetKwData{h}{HL}
\SetKwData{v}{VL}
\SetKwData{Hk}{Hk}
\SetKwData{Vk}{Vk}
\SetKwFunction{Bracket}{Bracket}%
\SetKwData{bracketPoly}{zavorkPoly}

\SetKwProg{Fn}{Function}{}{end}


\Fn{\Bracket{$L$}}{
\KwData{Diagram linku s $n$ kríženími}
\KwResult{Závorkový polynom v promenne $A$}
\If{link $L$ je kruznice}{\KwRet 1}
vyber \cross linku $L$
\\
\h $\leftarrow$ link $L$ , kde \cross je rozpojeno horizontálne
\\
\v $\leftarrow$ link $L$, kde \cross je rozpojeno vertikálne
\\

\eIf{v linku \h vznikla disjunktni kruznice} {Hk $\leftarrow$ 1} {Hk $\leftarrow$ 0}
\eIf{v linku \v vznikla disjunktni kruznice} {Vk $\leftarrow$ 1} {Vk $\leftarrow$ 0}

\bracketPoly $\leftarrow$ $A(-A^2 - A^{-2})^{\Hk} $ \Bracket{\h} + $A^{-1}(-A^2 - A^{-2})^{\Vk} $ \Bracket{\v}

\KwRet \bracketPoly
}

\caption{Závorkový polynom} 
\end{algorithm}

Závorkový polynom linku s $n$ kříženími se vypočte ze dvou závorkových polynomů linků s $n-1$ kříženími. Algoritmus má tedy časovou složitost $\mathcal{O}(2^n)$. Stejnou časovou složitost má i výpočet Jonesova polynomu používající tento postup.
Zastaví + koretnost.


\subsection{Průběžné rozmotávání}
Algoritmus na výpočet závorkového polynomu zrychlíme, pokud se link pokusíme v každém kroku rozmotat, tedy pokud nalezneme diagram ekvivalentního uzlu s menším množstvím křížení. 

V PD notaci jsou snadno naleznutelné případy, kdy lze link rozmotat použitím prvního či druhého Reidemastrova pohybu.

Při použití prvního Reidemastrova pohybu odmotáme jednu smyčkua zbavíme se jednoho křížení, ovšem výsledný závorkový polynom se změní o mocninu $A^3$ podle lemmatu.

Použitím druhého Reidematrova pohybu se zbavíme dvou křížení a polynom zůstane podle lemmatu stejný.

\begin{algorithm}[H]

\DontPrintSemicolon

\SetKwData{cross}{krizeni}
\SetKwData{h}{HL}
\SetKwData{v}{VL}
\SetKwData{Hk}{Hk}
\SetKwData{Vk}{Vk}
\SetKwData{L}{L}
\SetKwData{e}{e}
\SetKwFunction{Bracket}{Bracket}%
\SetKwData{bracketPoly}{zavorkPoly}

\SetKwProg{Fn}{Function}{}{end}


\Fn{\Bracket{\L}}{

Rozmotej link \L prvnim Reid pohybem\\
\If{neco rozmotano}{
\e $\leftarrow$ soucet znaminek rozmotanych krizeni\\
\KwRet $A^{3 \e}$ \Bracket{rozmotany link} \\
}
Rozmotej link \L druhym Reid pohybem\\
Jeste jednou rozmotej link \L prvnim Reid pohybem\\
\If{neco rozmotano}{
\e $\leftarrow$ soucet znaminek rozmotanych krizeni\\
\KwRet $A^{3 \e}$ \Bracket{rozmotany link} \\
}
Jeste kolik vzniklo samostatnych kruznic
.
.
.

\KwRet \bracketPoly
}

\caption{Závorkový polynom s rozmotáváním} 
\end{algorithm}

Zastaví + koretnost.
Rozmotávání běží v lineárním čase vzhledem k $n$, celková časová složitost tedy zůstává $\mathcal{O}(2^n)$.

\subsection{Vhodná volba křížení}
Tady nějak nezáleží na orientaci.

Algoritmus dále můžeme zlepšit vhodnou volbou křížení, které rozpojíme.
Dokážeme, že ke vzniku linku, který lze částečně rozmotat způsobem popsaným v předchozí části, je vždy potřeba rozpojit nejvýše dvě křížení.
Algoritmus bude volit právě ta křížení, jejichž rozpojení nám v dalších krocích zajistí možnost rozmotání.

\subsubsection{Diagram jako rovinný graf}
Každý linkový diagram odpovídá rovinnému grafu, v němž křížení představují vrcholy (vždy stupně čtyři) a úseky mezi kříženími hrany. Diagram s $n$ kříženími odpovídá grafu s $n$ vrcholy a 2$n$ hranami.
Dále budeme v této sekci k popisu diagramů používat grafovou terminologii. Předpokládejme také, že pracujeme s diagramem, který už je rozmotaný ve smyslu rozmotávání v sekci bla.

Eulerova formule pro rovinné grafy říká, že $v - e +f = 2$, kde $v$ značí počet vrcholů, $e$ počet hran a $f$ počet stěn.

V našem případě tedy dostáváme vzorec pro počet stěn $f = n+2$.

Každá hrana náleží dvěma stěnám, rozdělujeme tedy 4$n$ hran mezi $n+2$. Z toho plyne, že musí existovat stěna, která je ohraničená méně než čtyřmi hranami.

\subsubsection{Typy stěn}

Stěna s jednou hranou by v linku odpovídala smyčce, ty jsou ovšem podle předpokladu už odstraněny rozmotáním.

Stěna se dvěma hranami, která není rozmotatelná, musí v diagramu odpovídat jedné ze situací na obrázku. Všimněme si, že rozpojením křížení $a_1$ i $a_2$ buď ve vertikálním, nebo horizontálním směru vznikne smyčka. Volbou křížení $a_1$ nebo $a_2$ je tedy zaručeno, že jeden syn má po rozmotání nejvýše $n-2$ křížení.

Stěna se třemi hranami v diagramu odpovídá buď typu B, nebo C zobrazeným na obrázku.

Ve stěně typu B získáme horizontálním rozpojením křížení $b_1$ syna, jenž jde rozmotat druhým Reidematrovým pohybem. Jeho diagram tedy bude mít nejvýše $n-3$ křížení.

Ve stěně typu C není rozpojením žádného křížení rozmotání zaručeno. Ovšem rozpojením kterého koli křížení získáme jednoho syna se stěnou typu A. Existuje tedy prasyn s nejvýše $n-3$ kříženími.

Jelikož chceme maximalizovat rozmotání synů, jsou preference algoritmu na volbu křížení $b_1 < a_i  < c_i$. Z předchozího rozboru plyne, že alespoň jedno z těchto křížení musí existovat.

V PD notaci je možné toto křížení nalézt v lineárním čase vzhledem k $n$.

\subsection{Konečný algoritmus}
Pseudokód toho celého. 

Vždy se zjevně zastaví. Je korektní, bo to platilo pořád.
Každopádně tady uvést pseudokód.

\subsection{Implementace}
Python?
Práce, záludnosti? Rozdělování?
Rychlá práce s PD notací? Jak jsem toho docílila?
\section{Analýza složitosti algoritmu}
\subsection{Horní odhad}
Rychlost algoritmu je závislá na počtu křížení $n$ a na míře rozmotávávaní. Například na diagramu, který je můžné absolutně rozmotat použitím prvních dvou Reidemasterových pohybů poběží nejhůř v kvadratickém čase. Zdůvodnit?

Horní odhad rychlosti tedy provedeme na případu diagramu, v němž dojde k pouze k minimálnímu rozmotávání. Tedy nejdode k rozmotání jiných křížení než těch, která jsou zaručená rozborem v sekci Volba vhodného křížení.
Také v případech, kdy není předchozí volbou křížení zaručena existence stěny typu A, bude existovat pouze stěna typu C, která zaručuje rozmotání pouze u vnuka.

Průběh algoritmu zakreslíme jako binární strom, kde číslo ve vrcholu je počet křížení diagramu. V kořeni máme diagram s $n$ kříženími, v němž rozmotáme křížení stěny C, takže jeho synové jsou diagramy s $n-1$ kříženími obsahující stěnu typu A. Jejich synové budou tedy diagramy velikosti $n-2$ a $n-3$.
V lichých hladinách stromu má tedy vrchol syny o jedna menší, v sudých hladinách je jeden syn o dva menší.

\begin{forest}
  [n[n-1[n-2[n-3[n-4[...][...]][n-5[...][...]]][n-3[...][...]]][n-3[n-4[n-5[...][...]][n-6[...][...]]][n-4[...][...]]]][n-1[n-2[...][...]][n-3[...][...]]]]
\end{forest}

Získáme rekurentní rovnici na počet operací výpočtu závorkového polynomu diagramu velikosti $n$.

$$T(n) = 2 T(n-2) + 2 T(n-3) + C_L n + 2 C_L (n-1)$$

Člen $C_L n + 2 C_L (n-1)$, kde $C_L$ je konstanta, vyjadřuje, že na rozdělení každého diagramu na syny (rozmotání a volba křížení) je potřeba lineární počet operací vzhledem k počtu křížení.

Rekurence lze řešit pomocí vytvořujících funkcí a rozkladu na parciální zlomky. Výpočtem získáme vzorec
$$ T(n) \approx C_1 2^{0.82 n } + C_2 n 2^{0.09 n } + C_3 2^{0.09 n } + C_4 n + C_5 = \mathcal{O}( 2^{0.82 n }),  $$

kde $C_i$ jsou jisté konstanty.

Podle sekce je na výpočet Jonesova polynomu ze závorkového pouze lineární počet operací. Dohromady má tedy náš algoritmus na výpočet Jonesova polynomu časovou složitost $\mathcal{O}( 2^{0.82 n })$.

\subsection{Dolní odhad}
Kolečka.

Provedeme analýzu průběhu algoritmu na linku $L$ s diagramem, který vznikne z diagramu Borromeovských kruhů přidáváním kružnic. Kružnice přidáváme tak, aby vždy protla dva další kruhy celkem ve čtyřech bodech a vzniklý diagram byl alternující, tj. aby se střídala křížení vedená zdola a zvrchu. Příklad na obrázku.

Diagram skládající se z $k$ kružnic má $4k-6$ křížení.

Na obrázku je znázorněn možný průběh výpočtu závorkového polynomu na tomto diagramu takový, že jsou postupně odpojovány krajní kružnice. 

Z obrázku plyne, že časová složitost algoritmu na diagramu s $n$ kříženími splňuje rekurenci
$$ T(n) = 8T(n-4) + C_1 n + C_0  $$
pro jisté konstanty $C_0$, $C_1$.

A tedy $T(n) = \Omega(8^{\frac{n}{4}})  =  \Omega{O}(2^{0.75 n})$. Nebo i tady omega? To velké o je tu divné.

Dokázali jsme, že náš algoritmus na výpočet Jonesova polynomu má časovou složitost $\Omega(2^{0.75n})$.

