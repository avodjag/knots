%%% Druhá kapitola
\chapter{Výpočet Jonesova polynomu}


Je to v tride number P  
Popis algoritmu, vypocet horniho odhadu, dolní odhad pro nějakou třídu uzlů, na které se to rozbije, skripta z počítačové algebry, důkaz správnosti algoritmu
Odhad složitosti?


Bylo by zajímavé identifikovat, pro které typy uzlů je algoritmus efektivní a pro které naopak dosahuje nejhorších výsledků.
Rychle na kanonickych nakreslenich torus uzlu a preclikovych uzlu.

Asi citovat complexity zoo.
A třeba kapitoly?

Jak se používá velké O?
Už jsem viděla, že můžu použít O(0.83). Ale co s tím spodním?
Co problém, co algoritmus.

\section{Výpočetní složitost problému}

Podle (On the computational complexity of the Jones and Tutte polynomials) patří problém určení Jonesova polynomu alternujího uzlu do třídy složitosti \#P, dokonce je tento problém \# P-těžký. 

Třída \#P obsahuje problémy, jejichž cílem je určit počet přijímacích cest nedeterministického Turingova stroje, jedná se tedy o rozšíření problémů třídy NP. Například problém \# SAT znamená nejen určit, jestli existuje pravdivostní ohodnocení Boolovské formule, ale i spočítat, kolik takových ohodnocení existuje celkem.

Říct, co z toho plyne. Jako že nemůžu najít lineární algoritmus. Nebo bych jako byla fakt dobrá, kdyby ano.

\section{Algoritmus}
Nejdriv chcu popsat, že to tak jde. Pak dodat pseudokód.

Do jakých detailů? Jak popsat implementaci? Python? Důkaz správnosti - vždy se zastaví. Vstup s počtem křížení. Vstup je PD notace (to až nějak v implementaci).

--------------

Náš algoritmus dostane na vstupu diagram orientovaného linku s $n$ kříženími zapsaný v PD notaci.

\subsection{PD notace} 

PD notace je zápis sestávajací ze čtveřice čísel pro každé křížení a jednoznačně popisuje daný diagram. Zápis diagramu v PD notaci se získá následovně: úseky mezi kříženími se očíslují po směru orientace linku čísly od $1$ do $2 n$. Každé křížení se označí čtyřmi přilehlými úseky, přičemž se začne úsekem, který do křížení vstupuje spodem a pokračuje se s úseky proti směru hodinových ručiček. Viz obrázek.

\subsection{Přímočarý algoritmus}

Jak již bylo naznačeno, k výpočtu Jonesova polynomu používáme závorkový polynom a následnou normalizaci pomocí writhe.

Závorkový polynom lze spočítat jednoduchým rekurzivním algoritmem.

\begin{code}
Bracket(link)
   \textbf{if} link nemá křížení
       return 1
       
   vyber křížení.
   
   link_v := link s křížením rozpojeným vertikálně
   if link_v obsahuje disjukntní kružnici
       unknot_v := 1
   else
       unknot_v := 0
       
   link_h := link s křížením rozpojeným horizontálně
   
   if link_h obsahuje disjukntní kružnici
       unknot_h := 1
   else
       unknot_h := 0
  Tohle se na pseudokod vážně nehodí, nejde tady ani zvýrazňovat.
\end{code}

\begin{algorithm}
\KwData{Diagram v PD n otaci}
\KwResult{Závorkový polynom}
$\sqrt{3}$
\caption{How to write algorithms}
\end{algorithm}

Pak se k tomu přidá writhe (mám ukázat, jak se počítá z PD notace?) a je to.

Takže teď zápis algoritmu se substitucí a writhe.

Zjevně je tento algoritmus v $O(2^n)$.

\subsection{Průběžné rozmotávání}
Algoritmus můžeme zrychlit, pokud se ho pokusíme v každém kroku rozmotat. 

Odmotávání smyček odpovídá prvnímu Reidematrovi a musíme to pak přenásobit.

Také se můžeme zbavit dvou křížení druhým Reidematrovým pohybem.

Oba typy jsou rozmotatelné v lineárním čase vzhledem k počtu křížení.

\subsection{Vhodná volba křížení}
Tady nějak nezáleží na orientaci.

Algoritmu můžeme výrazně urychlit vhodnou volbou křížení k rozpojení ve výpočtu závorkového polynomu tak, aby bylo rozpojený uzel možné co nejvíce rozmotat.

Každý linkový diagram odpovídá rovinnému grafu, v němž křížení představují vrcholy (vždy stupně čtyři) a úseky mězi kříženími hrany. Dále budeme při popisu diagramů používat grafovou terminologii.

Link s $n$ kříženími odpovídá tedy grafu s $n$ vrcholy a 2$n$ hranami.

Eulerova formule pro rovinné grafy říká, že $v - e +f = 2$, kde $v$ značí počet vrcholů, $e$ počet hran a $f$ počet stěn.

V našem případě tedy dostáváme vzorec pro počet stěn $f = n+2$.

Každá hrana náleží dvěma stěnám, tedy rozdělujeme 4$n$ hran mezi $n+2$, takže musí existovat stěna, která je ohraničená méně než čtyřmi hranami.

Stěna s jednou hranou je právě smyčka. NOPE! ono je asi dulezite kvuli pocitani sten, ze je to rozmotane. NOPE~ v pohode

Stěna s dvěma hranami je buď odstranitelná, nebo typ A.

Stěna se třemi hranami je buď B1, nebo B2.

Typ je prostě nejlepší, ten vybereme a v dalším kroku hned půjde něco rozmotat.

Pak preferujeme typ B1 - není to blbost? Nemám nejdřív chtít B1?

Pak B2. Z něj uděláme B1 a v dalším kroku bude pohoda.

\subsection{Konečný algoritmus}
Pseudokód toho celého. Jones jako v tom prvním. Nebo ho uvést až tady a nahoře se na to odkázat.

Vždy se zjevně zastaví.
Každopádně tady uvést pseudokód.
\section{Analýza složitosti algoritmu}
\subsection{Horní odhad}
Algoritmus nejrychleji běží, pokud link rozmotává. K nejmenší míře rozmotávání dochází, pokud se tam vždycky najde jen B2. Musím si to rozdělit do sudého a lichého kroku. V sudém kroku nalezne jen B2, v lichém pak musí být B1, a tedy se jedna větev zmenší, ale jinak nic.

To se napíše do stromu.

Z toho se udělá rekurentní vzorec.

Z toho vypočítáme, jak to vychází.

Algoritmus má tedy časovou složitost $O(...)$

\subsection{Dolní odhad}
------------------------
Analýza jak to jde.

Dolní odhad
Kolečka.

Plyne z toho složitost?

Citovat PD notaci