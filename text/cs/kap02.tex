%%% Druhá kapitola
\chapter{Výpočet Jonesova polynomu}


Je to v tride number P  
Popis algoritmu, vypocet horniho odhadu, dolní odhad pro nějakou třídu uzlů, na které se to rozbije, skripta z počítačové algebry, důkaz správnosti algoritmu
Odhad složitosti?


Bylo by zajímavé identifikovat, pro které typy uzlů je algoritmus efektivní a pro které naopak dosahuje nejhorších výsledků.
Rychle na kanonickych nakreslenich torus uzlu a preclikovych uzlu.

Jak se používá velké O?
Už jsem viděla, že můžu použít O(0.83). Ale co s tím spodním?
Co problém, co algoritmus.

\section{Výpočetní složitost problému}

Podle (On the computational complexity of the Jones and Tutte polynomials) patří problém určení Jonesova polynomu alternujího uzlu do třídy složitosti \#P, dokonce je tento problém \# P-těžký. 

Třída \#P obsahuje problémy, jejichž cílem je určit počet přijímacích cest nedeterministického Turingova stroje, jedná se tedy o rozšíření problémů třídy NP. Například problém \# SAT znamená nejen určit, jestli existuje pravdivostní ohodnocení Boolovské formule, ale i spočítat, kolik takových ohodnocení existuje celkem.

Říct, co z toho plyne. Jako že nemůžu najít lineární algoritmus. Nebo bych jako byla fakt dobrá, kdyby ano.

\section{Algoritmus}
Nejdriv chcu popsat, že to tak jde. Pak dodat pseudokód.

Do jakých detailů? Jak popsat implementaci? Python? Důkaz správnosti - vždy se zastaví. Vstup s počtem křížení. Vstup je PD notace (to až nějak v implementaci).

--------------

Náš algoritmus dostane na vstupu diagram orientovaného linku s $n$ kříženími zapsaný v PD notaci.

\subsection{PD notace} 

PD notace je zápis sestávajací ze čtveřice čísel pro každé křížení a jednoznačně popisuje daný diagram. Zápis diagramu v PD notaci se získá následovně: úseky mezi kříženími se očíslují po směru orientace linku čísly od $1$ do $2 n$. Každé křížení se označí čtyřmi přilehlými úseky, přičemž se začne úsekem, který do křížení vstupuje spodem, a pokračuje se s úseky navazujícími proti směru hodinových ručiček. Viz obrázek.

\subsection{Výpočet Jonesova polynomu ze závorkového polynomu}

Jak již bylo řečeno, k výpočtu Jonesova polynomu používáme závorkový polynom. Podle věty se Jonesův polynom získá z normalizovaného závorkového polynomu substitucí proměnné. 

\begin{algorithm}[H]
\DontPrintSemicolon
\KwData{Diagram linku $L$  s $n$ krízení}
\KwResult{Jonesův polynom v proměnné $t$}
\SetKwData{bracketPoly}{závorkovýPolynom}
\SetKwData{writhe}{zamotání}
\SetKwData{jones}{jonesůvPolynom}
\SetKwData{normal}{normalizovanýPolynom}
\SetKwFunction{Writhe}{Writhe}
\SetKwFunction{Bracket}{Bracket}
\SetKwFunction{Substituce}{Substituce}

\BlankLine

\bracketPoly $\leftarrow$ \Bracket{$L$}
\tcc*[r]{v proměnné $A$}

\writhe $\leftarrow$ \Writhe{$L$}

\normal $\leftarrow$ $ (-A^3)^{\writhe} \times \bracketPoly $
\jones $\leftarrow$ \Substituce{\normal , $A$, $t^{1/2}$}

\KwRet \jones 

\caption{Jonesův polynom} 
\end{algorithm}

V PD notaci lze jednoduše určit, jestli je křížení kladné, či záporné orientace, tedy zamotání spočítáme v $\mathcal{O} (n)$ čase.
spočítáme v lineárním čase vzhledem k počtu křížení.

Dále se budeme zabývat výpočtem závorkový polynom.



\subsection{Přímočarý výpočet závorkového polynomu}
Z definice závorkového polynomu plyne jednoduchý rekurzivní algoritmus.

\begin{algorithm}[H]

\DontPrintSemicolon

\SetKwData{cross}{krizeni}
\SetKwData{h}{HL}
\SetKwData{v}{VL}
\SetKwData{Hk}{Hk}
\SetKwData{Vk}{Vk}
\SetKwFunction{Bracket}{Bracket}%
\SetKwData{bracketPoly}{zavorkPoly}

\SetKwProg{Fn}{Function}{}{end}


\Fn{\Bracket{$L$}}{
\KwData{Diagram linku s $n$ krízeními}
\KwResult{Závorkový polynom v promenne $A$}
\If{link $L$ je kruznice}{\KwRet 1}
vyber \cross linku $L$
\\
\h $\leftarrow$ link $L$ , kde \cross je rozpojeno horizontálne
\\
\v $\leftarrow$ link $L$, kde \cross je rozpojeno vertikálne
\\

\eIf{v linku \h vznikla disjunktni kruznice} {Hk $\leftarrow$ 1} {Hk $\leftarrow$ 0}
\eIf{v linku \v vznikla disjunktni kruznice} {Vk $\leftarrow$ 1} {Vk $\leftarrow$ 0}

\bracketPoly $\leftarrow$ $A(-A^2 - A^{-2})^{\Hk} $ \Bracket{\h} + $A^{-1}(-A^2 - A^{-2})^{\Vk} $ \Bracket{\v}

\KwRet \bracketPoly
}

\caption{Závorkový polynom} 
\end{algorithm}

Závorkový polynom linku s $n$ kříženími se vypočte ze dvou závorkových polynomů linků s $n-1$ kříženími. Algoritmus má tedy časovou složitost $O(2^n)$. Stejnou časovou složitost by měl i výpočet Jonesova polynomu používající tento postup.

\subsection{Průběžné rozmotávání}
Algoritmus na výpočet závorkového polynomu zrychlíme, pokud se link pokusíme v každém kroku rozmotat, tedy pokud nalezneme diagram ekvivalentního uzlu s menším množstvím křížení. 

V PD notaci jsou snadno naleznutelné případy, kdy lze link rozmotat použitím prvního či druhého Reidemastrova pohybu.

Při použití prvního Reidemastrova pohybu odmotáme jednu smyčkua zbavíme se jednoho křížení, ovšem výsledný závorkový polynom se změní o mocninu $A^3$.

Použitím druhého Reidematrova pohybu se zbavíme dvou křížení a polynom zůstane podle lemmatu stejný.

\begin{algorithm}[H]

\DontPrintSemicolon

\SetKwData{cross}{krizeni}
\SetKwData{h}{HL}
\SetKwData{v}{VL}
\SetKwData{Hk}{Hk}
\SetKwData{Vk}{Vk}
\SetKwFunction{Bracket}{Bracket}%
\SetKwData{bracketPoly}{zavorkPoly}

\SetKwProg{Fn}{Function}{}{end}


\Fn{\Bracket{$L$}}{
\KwData{Diagram linku s $n$ krízeními}
\KwResult{Závorkový polynom v promenne $A$}

.
.
.

\KwRet \bracketPoly
}

\caption{Závorkový polynom s rozmotáváním} 
\end{algorithm}


\subsection{Vhodná volba křížení}
Tady nějak nezáleží na orientaci.

Algoritmu můžeme výrazně urychlit vhodnou volbou křížení k rozpojení ve výpočtu závorkového polynomu tak, aby bylo rozpojený uzel možné co nejvíce rozmotat.

Každý linkový diagram odpovídá rovinnému grafu, v němž křížení představují vrcholy (vždy stupně čtyři) a úseky mězi kříženími hrany. Dále budeme při popisu diagramů používat grafovou terminologii.

Link s $n$ kříženími odpovídá tedy grafu s $n$ vrcholy a 2$n$ hranami.

Eulerova formule pro rovinné grafy říká, že $v - e +f = 2$, kde $v$ značí počet vrcholů, $e$ počet hran a $f$ počet stěn.

V našem případě tedy dostáváme vzorec pro počet stěn $f = n+2$.

Každá hrana náleží dvěma stěnám, tedy rozdělujeme 4$n$ hran mezi $n+2$, takže musí existovat stěna, která je ohraničená méně než čtyřmi hranami.

Stěna s jednou hranou je právě smyčka. NOPE! ono je asi dulezite kvuli pocitani sten, ze je to rozmotane. NOPE~ v pohode

Stěna s dvěma hranami je buď odstranitelná, nebo typ A.

Stěna se třemi hranami je buď B1, nebo B2.

Typ je prostě nejlepší, ten vybereme a v dalším kroku hned půjde něco rozmotat.

Pak preferujeme typ B1 - není to blbost? Nemám nejdřív chtít B1?

Pak B2. Z něj uděláme B1 a v dalším kroku bude pohoda.

\subsection{Konečný algoritmus}
Pseudokód toho celého. Jones jako v tom prvním. Nebo ho uvést až tady a nahoře se na to odkázat.

Vždy se zjevně zastaví.
Každopádně tady uvést pseudokód.
\section{Analýza složitosti algoritmu}
\subsection{Horní odhad}
Algoritmus nejrychleji běží, pokud link rozmotává. K nejmenší míře rozmotávání dochází, pokud se tam vždycky najde jen B2. Musím si to rozdělit do sudého a lichého kroku. V sudém kroku nalezne jen B2, v lichém pak musí být B1, a tedy se jedna větev zmenší, ale jinak nic.

To se napíše do stromu.

Z toho se udělá rekurentní vzorec.

Z toho vypočítáme, jak to vychází.

Algoritmus má tedy časovou složitost $O(...)$

\subsection{Dolní odhad}
------------------------
Analýza jak to jde.

Dolní odhad
Kolečka.

Plyne z toho složitost?

Citovat PD notaci