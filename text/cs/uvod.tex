\chapter*{Úvod}
\addcontentsline{toc}{chapter}{Úvod}

Matematický uzel je vnoření kružnice do trojrozměrného euklidovského prostoru, neboli neformálně zamotaný provázek se spojenými konci. Teorie uzlů se často zabývá otázkou, jak od sebe rozpoznat různé uzly či jak určit, jestli rozmotáním daného uzlu může vzniknout uzel triviální. Jedním způsobem, jak na tyto otázky odpovědět, je studium uzlových invariantů, tedy vlastností, které jsou stejné pro ekvivalentní uzly.
\\
Užitečným druhem invariantů jsou invarianty polynomiální, mezi něž patří například Alexanderův nebo	 Conwayův polynom. 
V roce 1985 publikoval novozélandský matematik Vaughan Jones práci o novém polynomu, který objevil při studiu operátorových algeber  \cite{jones1985}. 
Jeho výsledky měly velký vliv na vývoj oboru a objev nových druhů uzlových polynomů \cite{cromwell2004knots}.
\\
Tato práce se zabývá právě Jonesovým polynomem a algoritmem na jeho výpočet. Součástí práce je také implementace algoritmu a jeho testování.
\par 
Text je rozdělen do tří částí. \\
První kapitola je věnována definici Jonesova polynomu, jeho vlastnostem a ekvivalentní definici pomocí jiného uzlového polynomu, závorkového polynomu.
\\
V druhé kapitole je na základě poznatků kapitoly první odvozen algoritmus na výpočet Jonesova polynomu a jeho varianty. Je zde také odhadnuta jeho výpočetní složitost.
\\
V závěrečné kapitole jsou uvedeny výsledku testování algoritmu na datech. Algoritmus byl otestován mimo jiné na náhodných uzlech a lincích s větším počtem křížení a zvláštních typech uzlů. Jsou zde také srovnány rychlosti výpočtu různých variant algoritmu.

