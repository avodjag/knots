	%%% Hlavní soubor. Zde se definují základní parametry a odkazuje se na ostatní části. %%%

%% Verze pro jednostranný tisk:
% Okraje: levý 40mm, pravý 25mm, horní a dolní 25mm
% (ale pozor, LaTeX si sám přidává 1in)
%hidelinks
\documentclass[12pt,a4paper]{report}
\setlength\textwidth{145mm}
\setlength\textheight{247mm}
\setlength\oddsidemargin{15mm}
\setlength\evensidemargin{15mm}
\setlength\topmargin{0mm}
\setlength\headsep{0mm}
\setlength\headheight{0mm}
% \openright zařídí, aby následující text začínal na pravé straně knihy
\let\openright=\clearpage

%% Pokud tiskneme oboustranně:
% \documentclass[12pt,a4paper,twoside,openright]{report}
% \setlength\textwidth{145mm}
% \setlength\textheight{247mm}
% \setlength\oddsidemargin{14.2mm}
% \setlength\evensidemargin{0mm}
% \setlength\topmargin{0mm}
% \setlength\headsep{0mm}
% \setlength\headheight{0mm}
% \let\openright=\cleardoublepage

%% Vytváříme PDF/A-2u
\usepackage[a-2u]{pdfx}

%% Přepneme na českou sazbu a fonty Latin Modern
\usepackage[czech]{babel}
\usepackage{lmodern}
\usepackage[T1]{fontenc}
\usepackage{textcomp}

%% Použité kódování znaků: obvykle latin2, cp1250 nebo utf8:
\usepackage[utf8]{inputenc}

%%% Další užitečné balíčky (jsou součástí běžných distribucí LaTeXu)
\usepackage{amsmath}        % rozšíření pro sazbu matematiky
\usepackage{amsfonts}       % matematické fonty
\usepackage{amsthm}         % sazba vět, definic apod.
\usepackage{bbding}         % balíček s nejrůznějšími symboly
			    % (čtverečky, hvězdičky, tužtičky, nůžtičky, ...)
\usepackage{bm}             % tučné symboly (příkaz \bm)
\usepackage{graphicx}       % vkládání obrázků
\usepackage{fancyvrb}       % vylepšené prostředí pro strojové písmo
\usepackage{indentfirst}    % zavede odsazení 1. odstavce kapitoly
%\usepackage{natbib}         % zajištuje možnost odkazovat na literaturu
			    % stylem AUTOR (ROK), resp. AUTOR [ČÍSLO]
\usepackage[nottoc]{tocbibind} % zajistí přidání seznamu literatury,
                            % obrázků a tabulek do obsahu
\usepackage{icomma}         % inteligetní čárka v matematickém módu
\usepackage{dcolumn}        % lepší zarovnání sloupců v tabulkách
\usepackage{booktabs}       % lepší vodorovné linky v tabulkách
\usepackage{paralist}       % lepší enumerate a itemize
\usepackage[usenames]{xcolor}  % barevná sazba

% mnou pridane
\usepackage{ amssymb }
\usepackage[ruled, czech, vlined]{algorithm2e} 
\usepackage[linguistics]{forest}
\usepackage{subcaption}
\usepackage{float}
%\usepackage{breqn}


\usepackage{tikz}
\usetikzlibrary{
  knots,
  hobby,
  decorations.pathreplacing,
  shapes.geometric,
  calc,
  decorations.markings,
  bending
}
%geogebra
\usepackage{pgf,pgfplots}
\pgfplotsset{compat=1.15}
\usepackage{mathrsfs}
\usetikzlibrary{arrows}
%\usepackage{pgf,pgfplots}
%\usepackage{mathrsfs}
%\usetikzlibrary{arrows}
%\usepackage{algorithmic} 

%%% Údaje o práci

% Název práce v jazyce práce (přesně podle zadání)
\def\NazevPrace{Jonesův polynom}

% Název práce v angličtině
\def\NazevPraceEN{Jones polynomial}

% Jméno autora
\def\AutorPrace{Anna Gajdová}

% Rok odevzdání
\def\RokOdevzdani{2018}

% Název katedry nebo ústavu, kde byla práce oficiálně zadána
% (dle Organizační struktury MFF UK, případně plný název pracoviště mimo MFF)
\def\Katedra{Katedra algebry}
\def\KatedraEN{Department of Algebra}

% Jedná se o katedru (department) nebo o ústav (institute)?
\def\TypPracoviste{Katedra}
\def\TypPracovisteEN{Department}

% Vedoucí práce: Jméno a příjmení s~tituly
\def\Vedouci{doc. RNDr. Stanovský David, Ph.D.}

% Pracoviště vedoucího (opět dle Organizační struktury MFF)
\def\KatedraVedouciho{Katedra algebry}
\def\KatedraVedoucihoEN{Department of Algebra}

% Studijní program a obor
\def\StudijniProgram{Matematika}
\def\StudijniObor{obecná matematika}

% Nepovinné poděkování (vedoucímu práce, konzultantovi, tomu, kdo
% zapůjčil software, literaturu apod.)
\def\Podekovani{%
Chtěla bych tímto poděkovat doc. Davidu Stanovskému za odborné vedení této práce. 
Také bych chtěla poděkovat své rodině za podporu během mého studia.
}

% Abstrakt (doporučený rozsah cca 80-200 slov; nejedná se o zadání práce)
\def\Abstrakt{%
Tématem této práce je Jonesův polynom daného uzlu a jeho výpočet. Nejprve definujeme Jonesův polynom dvěma způsoby: pomocí skein vztahů a~pomocí závorkového polynomu a dokážeme ekvivalenci těchto definic. Dále na základě vztahu Jonesova a závorkového polynomu odvodíme algoritmus na jeho výpočet. Dokážeme, že algoritmus má časovou složitost $\mathcal{O}\left(2^{0,823n}\right)$, kde $n$ značí počet křížení linkového diagramu. Nakonec shrneme výsledky testování algoritmu a jeho variant na datech. Algoritmus otestujeme mimo jiné na  malých tabulkových uzlech, větších náhodných uzlech a torusových uzlech. U~nejrychlejší varianty algoritmu odhadneme průměrnou časovou složitost výpočtu na náhodných uzlech $\mathcal{O}\left(2^{0,487 n+ o(n)}\right)$.
}
\def\AbstraktEN{%
The topic of this thesis is the Jones polynomial of a given knot and its computation. First we define the Jones polynomial in two ways: using skein relations and using the bracket polynomial and we prove that these definitions are equivalent. Next we derive an algorithm for computation of the Jones polynomial based on its relation with the bracket polynomial. We prove that the time complexity of the algorithm is $\mathcal{O}\left(2^{0.823n}\right)$, where $n$ denotes number of crossings in a link diagram. Lastly we present the results of running the algorithm and its variants on data. We test the algorithm among others on small table knots, bigger random knots and on torus knots. We estimate that the fastest variant of the algorithm runs on random knots with the average time complexity $\mathcal{O}\left(2^{0.487 n+ o(n)}\right)$.
}

% 3 až 5 klíčových slov (doporučeno), každé uzavřeno ve složených závorkách
\def\KlicovaSlova{%
{teorie uzlů}, {Jonesův polynom}, {uzlový invariant}
}
\def\KlicovaSlovaEN{%
{knot theory}, {Jones polynomial}, {knot invariant}
}

%% Balíček hyperref, kterým jdou vyrábět klikací odkazy v PDF,
%% ale hlavně ho používáme k uložení metadat do PDF (včetně obsahu).
%% Většinu nastavítek přednastaví balíček pdfx.
\hypersetup{unicode}
\hypersetup{breaklinks=true}

%% Definice různých užitečných maker (viz popis uvnitř souboru)
%%% Tento soubor obsahuje definice různých užitečných maker a prostředí %%%
%%% Další makra připisujte sem, ať nepřekáží v ostatních souborech.     %%%

%%% Drobné úpravy stylu

% Tato makra přesvědčují mírně ošklivým trikem LaTeX, aby hlavičky kapitol
% sázel příčetněji a nevynechával nad nimi spoustu místa. Směle ignorujte.
\makeatletter
\def\@makechapterhead#1{
  {\parindent \z@ \raggedright \normalfont
   \Huge\bfseries \thechapter. #1
   \par\nobreak
   \vskip 20\p@
}}
\def\@makeschapterhead#1{
  {\parindent \z@ \raggedright \normalfont
   \Huge\bfseries #1
   \par\nobreak
   \vskip 20\p@
}}
\makeatother

% Toto makro definuje kapitolu, která není očíslovaná, ale je uvedena v obsahu.
\def\chapwithtoc#1{
\chapter*{#1}
\addcontentsline{toc}{chapter}{#1}
}

% Trochu volnější nastavení dělení slov, než je default.
\lefthyphenmin=2
\righthyphenmin=2

% Zapne černé "slimáky" na koncích řádků, které přetekly, abychom si
% jich lépe všimli.
\overfullrule=1mm

%%% Makra pro definice, věty, tvrzení, příklady, ... (vyžaduje baliček amsthm)

\theoremstyle{plain}
\newtheorem{veta}{Věta}
\newtheorem{lemma}[veta]{Lemma}
\newtheorem{tvrz}[veta]{Tvrzení}

\theoremstyle{plain}
\newtheorem{definice}{Definice}

\theoremstyle{remark}
\newtheorem*{dusl}{Důsledek}
\newtheorem*{pozn}{Poznámka}
\newtheorem*{prikl}{Příklad}

%%% Prostředí pro důkazy

\newenvironment{dukaz}{
  \par\medskip\noindent
  \textit{Důkaz}.
}{
\newline
\rightline{$\square$}  % nebo \SquareCastShadowBottomRight z balíčku bbding
}

%%% Prostředí pro sazbu kódu, případně vstupu/výstupu počítačových
%%% programů. (Vyžaduje balíček fancyvrb -- fancy verbatim.)

\DefineVerbatimEnvironment{code}{Verbatim}{fontsize=\small, frame=single}

%%% Prostor reálných, resp. přirozených čísel
\newcommand{\R}{\mathbb{R}}
\newcommand{\N}{\mathbb{N}}
% ja
\newcommand{\Z}{\mathbb{Z}}

%%% Užitečné operátory pro statistiku a pravděpodobnost
\DeclareMathOperator{\pr}{\textsf{P}}
\DeclareMathOperator{\E}{\textsf{E}\,}
\DeclareMathOperator{\var}{\textrm{var}}
\DeclareMathOperator{\sd}{\textrm{sd}}

%%% Příkaz pro transpozici vektoru/matice
\newcommand{\T}[1]{#1^\top}

%%% Vychytávky pro matematiku
\newcommand{\goto}{\rightarrow}
\newcommand{\gotop}{\stackrel{P}{\longrightarrow}}
\newcommand{\maon}[1]{o(n^{#1})}
\newcommand{\abs}[1]{\left|{#1}\right|}
\newcommand{\dint}{\int_0^\tau\!\!\int_0^\tau}
\newcommand{\isqr}[1]{\frac{1}{\sqrt{#1}}}

%%% Vychytávky pro tabulky
\newcommand{\pulrad}[1]{\raisebox{1.5ex}[0pt]{#1}}
\newcommand{\mc}[1]{\multicolumn{1}{c}{#1}}

\newcommand{\minuskriz}{
\begin{tikzpicture}[scale=0.4, baseline=2]
\draw [ultra thick]   (0,0) -- (0.33,0.33);
\draw [ultra thick]    (0.66,0.66)-- (1,1);
\draw [ultra thick]   (0,1)  -- (1,0);
\end{tikzpicture}
}

\newcommand{\kruh}{
\begin{tikzpicture}[scale=0.4, baseline=-5pt]
\draw [ultra thick]   (0,0) circle (0.5cm);
\end{tikzpicture}
}

\newcommand{\pluskriz}{\begin{tikzpicture}[scale=0.4, baseline=2]
\draw [ultra thick]   (0,0) -- (1,1);
\draw [ultra thick]   (0,1) -- (0.33, 0.66);
\draw [ultra thick]   (0.66, 0.33) -- (1,0);
\end{tikzpicture}}

\newcommand{\vertkriz}{\begin{tikzpicture} [scale=0.4, baseline=2]
\draw [ultra thick]   (0,0) .. controls (1/2,1/2)  .. (0,1);
\draw [ultra thick]   (1,0) .. controls (1/2,1/2)  .. (1,1);
\end{tikzpicture}}

\newcommand{\horkriz}{\begin{tikzpicture} [scale=0.4, baseline=2]
\draw [ultra thick]   (0,0) .. controls (1/2,1/2)  .. (1,0);
\draw [ultra thick]   (0,1) .. controls (1/2,1/2)  .. (1,1);
\end{tikzpicture}}

\newcommand{\plussmycka}{
\begin{tikzpicture}[baseline =\dimexpr-\fontdimen22\textfont2, scale = 0.25]
\begin{knot}[
consider self intersections=true,
clip width = 4,
end tolerance=1pt,
] 
\clip (0,-1) rectangle (2.5,1);
\strand [ultra thick]   (0,-1) .. controls (3,3) and (3,-3) ..  (0,1);
\end{knot}
\end{tikzpicture}}

\newcommand{\odsmycka}{
\begin{tikzpicture}[baseline =\dimexpr-\fontdimen22\textfont2,scale = 0.25]
\clip (0,-1) rectangle (2,1);
\draw [ultra thick]   (0,1) .. controls (2.5,0) ..  (0,-1);
\end{tikzpicture}}

\newcommand{\minussmycka}{
\begin{tikzpicture}[baseline =\dimexpr-\fontdimen22\textfont2,scale = 0.25]
\begin{knot}[
consider self intersections=true,
clip width = 4,
end tolerance=1pt,
] 
\clip (0,-1) rectangle (2.5,1);
\strand [ultra thick]   (0,1) .. controls (3,-3) and (3,3) ..  (0,-1);
\end{knot}
\end{tikzpicture}}

\newcommand{\AO}{
\begin{tikzpicture}[scale =0.3, baseline = -4]
\draw [ultra thick]  (0,-1) arc (270:90:1);
\draw [ultra thick]  (0, 1) .. controls (1,1) .. (1, 1.5); 
\draw [ultra thick]  (1, -1.2) .. controls (1,1) .. (2, 1); 
\draw [ultra thick]  (0, -1) -- (0.75, -1);
\draw [ultra thick]  (1.25, -1) -- (2, -1);

\end{tikzpicture}}


\newcommand{\BO}{
\begin{tikzpicture}[scale =0.3, baseline = -4]
\draw [ultra thick]  (0,-1) arc (270:90:1);
\draw [ultra thick]  (0, 1) .. controls (1,1) .. (1, 1.5); 
\draw [ultra thick]  (1, -1.2) .. controls (1,1) .. (2, 1); 
\draw [ultra thick]  (0, -1) -- (0.75, -1);
\draw [ultra thick]  (1.25, -1) -- (2, -1);

\end{tikzpicture}}


\newcommand{\CO}{
\begin{tikzpicture}[scale =0.3, baseline = -4]
\draw [ultra thick]  (4, -1.3) -- (4, 1.3);


\draw [ultra thick]  (5.5,-1) arc (270:90:1);
\draw [ultra thick]  (5.5,-1) -- (7,-1);
\draw [ultra thick]  (5.5,1) -- (7,1);

\end{tikzpicture}}


\newcommand{\DO}{
\begin{tikzpicture}[scale =0.3, baseline = -4]

\draw [ultra thick]  (1, 1.5) .. controls (1,0.5) .. (3, 0.5);
\draw [ultra thick]  (1, -1.5) .. controls (1,-0.5) .. (3, -0.5);


\end{tikzpicture}}


\newcommand{\EO}{
\begin{tikzpicture}[scale =0.3, baseline = -4]
\draw [ultra thick]  (0,0) circle (0.4cm);

\draw [ultra thick]  (1, 1.5) .. controls (1,0.5) .. (3, 0.5);
\draw [ultra thick]  (1, -1.5) .. controls (1,-0.5) .. (3, -0.5);
\end{tikzpicture}}


\newcommand{\FO}{
\begin{tikzpicture}[scale =0.3, baseline = -4]
\clip (-0.75, -1.5) rectangle (3, 1.5);

\draw [ultra thick] (1, 1.5) .. controls (1,0.5) .. (3, 0.5);

\draw [ultra thick] (0.75, -0.5) .. controls (-2, -0.5) and (1, 2) .. (1, -1.5);
\draw [ultra thick] (1.25, -0.5) -- (3, -0.5);
\end{tikzpicture}}

\newcommand{\reiddva}{
\begin{tikzpicture}[scale =0.3, baseline = -4]

\draw [ultra thick]  (-1,-1) arc (270:90:1);
%\draw (-1, -1) -- (1, -1);
\draw [ultra thick]  (-1, -1) -- (-0.25, -1);
\draw [ultra thick]  (0.25, -1) -- (1, -1);
%\draw (-1, 1) -- (1, 1);
\draw [ultra thick]  (0, -1.3) -- (0, 1.3);

\draw [ultra thick]  (-1, 1) -- (-0.25, 1);
\draw [ultra thick]  (0.25, 1) -- (1, 1);
\end{tikzpicture}}

\newcommand{\GO}{
\begin{tikzpicture}[scale =0.3, baseline = -4]
\draw [ultra thick]  (1,-1) -- (2.2, -0.2);
\draw [ultra thick]  (2.8, 0.2) -- (3.4, 0.6);
\draw [ultra thick]  (3.58, 0.72) -- (4, 1);
\draw [ultra thick]  (1,1) --  (3.4, -0.6) ;
\draw [ultra thick]  (3.58, -0.72) -- (4, -1);

\draw (3.5,-1) -- (3.5, 1);
\end{tikzpicture}}

\newcommand{\HO}{
\begin{tikzpicture}[scale =0.3, baseline = -4]
\draw [ultra thick]  (-4,1) -- (-3.58, 0.72);
\draw [ultra thick]  (-3.4, 0.6) -- (-1,-1);
\draw [ultra thick]  (-4, -1) -- (-3.58, -0.72);
\draw [ultra thick]  (-3.4, -0.6) -- (-2.8, -0.2);
\draw [ultra thick]  (-2.2, 0.2) -- (-1, 1);
\draw [ultra thick]  (-3.5, -1) -- (-3.5, 1);
\end{tikzpicture}}

\newcommand{\IO}{
\begin{tikzpicture}[scale =0.3, baseline = -4]

\draw [ultra thick]  (-1,-1) arc (270:90:1);
%\draw (-1, -1) -- (1, -1);
\draw [ultra thick]  (-1, -1) -- (-0.25, -1);
\draw [ultra thick]  (0.25, -1) -- (1, -1);
%\draw (-1, 1) -- (1, 1);
\draw [ultra thick]  (0, -1.3) -- (0, 1.3);

\draw [ultra thick]  (-1, 1) -- (-0.25, 1);
\draw [ultra thick]  (0.25, 1) -- (1, 1);

\draw [ultra thick]  (-2.25, -1) -- (-2.25, 1);
\end{tikzpicture}}

\newcommand{\JO}{
\begin{tikzpicture}[scale =0.3, baseline = -4]

\draw [ultra thick]  (3.4,-1) -- (3.4, 1);
\draw [ultra thick]  (4, 0.8) -- (3.7,0.8);
\draw [ultra thick]  (3.1,0.8) -- (1.5, 0.8);
\draw [ultra thick]  (4, -0.8) -- (3.7,-0.8);
\draw [ultra thick]  (3.1,-0.8) -- (1.5, -0.8);
\end{tikzpicture}}

\newcommand{\KO}{
\begin{tikzpicture}[scale =0.3, baseline = -4]

\draw [ultra thick]  (-1,-1) arc (270:90:1);
\draw [ultra thick]  (-1, -1) -- (1, -1);
\draw [ultra thick]  (-1, 1) -- (1, 1);

\draw [ultra thick]  (-1, 1) -- (-0.25, 1);
\draw [ultra thick]  (0.25, 1) -- (1, 1);

\draw [ultra thick]  (-2.25, -1) -- (-2.25, 1);
\draw [ultra thick]  (-2.6, -1) -- (-2.6, 1);

\end{tikzpicture}}

\newcommand{\LO}{
\begin{tikzpicture}[scale =0.3, baseline = -4]

%\draw (1,-1) arc (0:90:1);
\draw [ultra thick]  (1,-1) arc (-90:90:1);
\draw [ultra thick]  (1, -1) -- (0.25, -1);
\draw [ultra thick]  (-0.25, -1) -- (-1, -1);
\draw [ultra thick]  (-0, -1.3) -- (-0, 1.3);

\draw [ultra thick]  (1, 1) -- (0.25, 1);
\draw [ultra thick] (-0.25, 1) -- (-1, 1);

\draw [ultra thick] (2.25, -1) -- (2.25, 1);
\end{tikzpicture}}

%% Titulní strana a různé povinné informační strany
\begin{document}
\include{titulka}

%%% Strana s automaticky generovaným obsahem bakalářské práce

\tableofcontents

%%% Jednotlivé kapitoly práce jsou pro přehlednost uloženy v samostatných souborech
\chapter*{Úvod}
\addcontentsline{toc}{chapter}{Úvod}

Studium uzlových invariantů a polynomů

co v které kapitole

Zmínit ty první dvě knihy
A článek 
A tak



%%% První kapitola

\chapter{Definice a vlastnosti Jonesova polynomu}
Cílem této kapitoly je definovat Jonesův polynom, dokázat jeho základní vlastnosti a popsat souvislost se závorkovým polynomem. Vycházíme z materiálů~\cite{cromwell2004knots, Adams2004, jones2005, weiping2016lecture} a vypracování cvičení z těchto zdrojů.
\section{Základní pojmy}
Matematický \emph{uzel} je vnoření kružnice do trojrozměrného euklidovského prostoru. Dva uzly jsou \emph{ekvivalentní}, pokud mezi nimi existuje homeomorfismus zachovávající orientaci. \emph{Invariant} je zobrazení, které ekvivalentním uzlům přiřadí stejnou hodnotu.
\emph{Link} je více disjunktních či propletených uzlů. Pokud není řečeno jinak, pracujeme v textu s linky. 
\\
Při definování Jonesova polynomu je důležité rozlišovat mezi linkem a jeho \emph{diagramem}. Diagram je vhodné rovinné nakreslení určité linkové projekce, v němž je rozlišeno, jestli křížení vedou \emph{svrchu}, nebo \emph{zdola}.  \emph{Triviální uzel} je uzel, k němuž existuje diagram bez křížení.
\\
V diagramu \emph{orientovaného} linku rozlišujeme křížení s \emph{kladnou} a se \emph{zápornou orientací}, viz obrázek~\ref{orientace}.

\begin{figure}[h]    

\centering 
\begin{subfigure}[t]{0.4\linewidth}\centering
\begin{tikzpicture}[scale=1.6] 
\draw [thick] (0,0) -- (0.33,0.33);
\draw [thick,->] (0.66,0.66)-- (1,1);
\draw [thick,->] (0,1)  -- (1,0);
\end{tikzpicture} 
\caption{Kladná orientace.} 
\end{subfigure}
\begin{subfigure}[t]{0.4\linewidth}\centering
\begin{tikzpicture}[scale=1.6]
\draw [thick,->] (0,0) -- (1,1);
\draw [thick] (0,1) -- (0.33, 0.66);
\draw [thick,->] (0.66, 0.33) -- (1,0);
\end{tikzpicture}  
\caption{Záporná orientace.}
\end{subfigure}

\caption{Orientace křížení.}
\label{orientace}
\end{figure}  


Pro popis polynomů na uzlech a lincích se často používají \emph{skein vztahy} \footnote{česky přadenové vztahy}.
Skein vztahy určují, jaká je spojitost mezi polynomy tří linků $L_+$, $ L_-$ a $L_0$, jejichž diagramy jsou identické až na oblast jednoho křížení. V linku $L_+$ má toto křížení kladnou orientaci, v $L_-$ zápornou a v $L_0$ je křížení rozpojené, viz obrázek~\ref{skeinobr}.

\begin{figure}[h]    
\centering 
\begin{subfigure}[t]{0.4\linewidth}\centering
\begin{tikzpicture}[scale=1.6] 
\draw [thick] (0,0) -- (0.33,0.33);
\draw [thick,->] (0.66,0.66)-- (1,1);
\draw [thick,->] (0,1)  -- (1,0);
\end{tikzpicture} 
\caption{$L_+$} 
\end{subfigure}
\begin{subfigure}[t]{0.4\linewidth}\centering
\begin{tikzpicture}[scale=1.6]
\draw [thick,->] (0,0) -- (1,1);
\draw [thick] (0,1) -- (0.33, 0.66);
\draw [thick,->] (0.66, 0.33) -- (1,0);
\end{tikzpicture}  
\caption{$L_-$}
\end{subfigure}
\begin{subfigure}[t]{0.4\linewidth}\centering
\begin{tikzpicture} [scale=1.6]
\draw  [thick,->](0,0) .. controls (1/2,1/2)  .. (1,0);
\draw  [thick,->](0,1) .. controls (1/2,1/2)  .. (1,1);
\end{tikzpicture}
\caption{$L_0$}
\end{subfigure}
\caption{Diagramy skein vztahu.}
\label{skeinobr}
\end{figure}

\section{Definice Jonesova polynomu}

\begin{definice}\label{def01:1}
\emph{Jonesův polynom} orientovaného linku $L$ je Laurentův polynom v~proměnné $\sqrt{t}$ (tj. polynom v $\Z\left[t^{1/2}, t^{-1/2}\right]$), značený $V_L(t)$ , který
\begin{enumerate}
\item
je linkový invariant,
\item 
  je normalizovaný, tedy polynom  $V_{\pmb{\circlearrowleft}}$ =1, kde ${\pmb{\circlearrowleft}}$ značí orientovaný triviální uzel,
\item  
splňuje skein vztah 
\begin{equation} \label{skein}
\frac{1}{t} V_{L_+} - t V_{L_-} = \left( \sqrt{t}  - \frac{1}{\sqrt{t}}\right) V_{L_0}.
\end{equation}
\end{enumerate}
\end{definice}

\begin{lemma}\label{l01:1}
Buď $L$ link, který se skládá z $k$ neprotínajících se orientovaných triviálních uzlů. Pak pro Jonesův polynom linku $L$ platí $$V_L(t) = \left(- \sqrt{t} -\frac{1}{\sqrt{t}}\right) ^{k-1}.$$
\end{lemma}
\begin{dukaz}
Libovolně orientované triviální uzly jsou navzájem ekvivalentní. Vzorec tedy stačí dokázat pro diagram skládající se z $k$ libovolně orientovaných disjunktních kružnic. Použijeme matematickou indukci.\\
Pro $ k=1$ vzorec platí podle druhé podmínky v definici ~\ref{def01:1}. \\
Předvedeme i případ, kdy $k = 2$. Pak $L_0 = $
\begin{tikzpicture}[line cap=round,line join=round,>=triangle 45,x=1cm,y=1cm, scale = 0.2, baseline=-3]
\clip(-1.1,-1.1) rectangle (3.5,1.1);
\draw [thick] (0,0) circle (1cm);
\draw [thick] (2.38,0) circle (1cm);
\draw [thick][line width=1/2pt] (0.6246950475544243,0.7808688094430303)-- (1,0.79);
\draw [thick][line width=1/2pt] (0.6246950475544243,0.7808688094430303)-- (0.58,0.39);
\draw [thick][line width=1/2pt] (1.7783415438306172,0.7987534676731457)-- (1.42,0.77);
\draw [thick][line width=1/2pt] (1.7783415438306172,0.7987534676731457)-- (1.8,0.41);
\end{tikzpicture}, $L_- = $
\begin{tikzpicture}[line cap=round,line join=round,>=triangle 45,x=1cm,y=1cm, scale = 0.2, baseline = -3]
\clip(-1.1,-1.1) rectangle (3.196466376386043,1.0990637767762312);
\draw [thick]  [shift={(0,0)}]   plot[domain=0:5.105410206761974,variable=\t]({1*1*cos(\t r)+0*1*sin(\t r)},{0*1*cos(\t r)+1*1*sin(\t r)});
\draw [thick]  [shift={(2,0)}]  plot[domain=-3.141592653589793:1.9340463984318206,variable=\t]({1*1*cos(\t r)+0*1*sin(\t r)},{0*1*cos(\t r)+1*1*sin(\t r)});
\draw [thick]  [line width=0.5pt] (0.6222441255753678,0.7828232547561078)-- (1,0.78);
\draw [thick]  [line width=0.5pt] (0.64,0.4)-- (0.6222441255753678,0.7828232547561078);
\end{tikzpicture}
 a $L_+ = $
\begin{tikzpicture}[line cap=round,line join=round,>=triangle 45,x=1cm,y=1cm, scale = 0.2, baseline =-3]
\clip(-1.1403462213720332,-1.2) rectangle (3.0406984599653097,1.225538970539465);
\draw [thick]  [shift={(0,0)}]  plot[domain=-5.158880748950075:0,variable=\t]({1*1.0280466904740395*cos(\t r)+0*1.0280466904740395*sin(\t r)},{0*1.0280466904740395*cos(\t r)+1*1.0280466904740395*sin(\t r)});
\draw [thick]  [shift={(2,0)}]  plot[domain=-2.0668400051610742:3.141592653589793,variable=\t]({1*0.9664884893261791*cos(\t r)+0*0.9664884893261791*sin(\t r)},{0*0.9664884893261791*cos(\t r)+1*0.9664884893261791*sin(\t r)});
\draw [thick]  [line width=0.5pt] (1.9692781611446357,0.6947891351587211)-- (1.977324654938205,0.966222453023282);
\draw [thick]  [line width=0.5pt] (1.7119284298490884,1.200910273373297)-- (1.977324654938205,0.966222453023282);
\end{tikzpicture}. Diagramy $L_+$ a $L_-$ zobrazují triviální uzly, takže $V_{L_+} = V_{L_-} = 1$. Použitím skein vztahu získáme $$V_ L = V_{L_0} = - \sqrt{t} -\frac{1}{\sqrt{t}} .$$ \\
Pro $k > 2$ jsou $L_-$ a $L_+ $ diagramy linků s $k-1$ kružnicemi. Z indukčního předpokladu a ze skein vztahu získáme vzorec $$V_ L = V_{L_0} = \left(- \sqrt{t} -\frac{1}{\sqrt{t}}\right) ^{k-1}.$$
\end{dukaz}  

\begin{pozn}
Z každého uzlového diagramu lze změnou několika křížení vedených shora na křížení vedených zdola získat diagram triviálního uzlu~\cite{Adams2004}. Z každého diagramu linku tedy můžeme změnou křížení získat diagram sjednocení triviálních uzlů, jejichž Jonesův polynom je podle předchozího lemmatu známý. Jonesův polynom každého linku lze tedy pomocí skein vztahu rekurzivně spočítat z jeho libovolného diagramu. Z toho plyno korektnost a jednoznačnost definice.
\end{pozn}


Definice Jonesova polynomu pomocí skein vztahů není vhodná pro algoritmický výpočet, neboť rozpoznat, jestli diagram odpovídá triviálnímu uzlu, je složitý problém. K výpočtu využijeme ekvivalentní definici založenou na použití \emph{závorkového polynomu}.

\section{Závorkový polynom}
Závorkový polynom \footnote{anglicky bracket polynomial nebo Kauffman bracket} je definován pouze pro diagramy neorientovaných linků, nikoli pro samotné linky.

\begin{definice}\label{def01:2}
\emph{Závorkový polynom} neorientovaného diagramu $D$, značený $\langle D \rangle$, je Laurentův polynom v proměnné $A$ definovaný třemi odvozovacími pravidly:
\begin{enumerate}
\item
$ \left\langle \pmb{\bigcirc} \right\rangle = 1$, kde $\pmb{\bigcirc}$ značí diagram s jednou komponentou bez křížení,
\item
$ \left\langle  \pluskriz
\right\rangle = A  \left\langle 
%vert
\vertkriz
 \right\rangle + A^{-1}  \left\langle
%hor 
\horkriz
\right\rangle $, kde
\begin{itemize} \item ~~\pluskriz~značí~~diagram obsahující toto křížení, \item ~~\vertkriz~~je diagram, který je s ním shodný až na dané křížení, které je zde \emph{vertikální rozpojeno}, \item ~~\horkriz~~je diagram, v němž je křížení \emph{rozpojeno horizontálně}, \end{itemize}
\item
$ \left\langle D \cup \pmb{\bigcirc} \right\rangle = \left(-A^2 - A^{-2}\right) \left\langle D \right\rangle$, kde $D \cup \pmb{\bigcirc} $ značí sjednocení diagramu $D$ a~diagramu s jednou komponentou bez křížení.
\end{enumerate}
\end{definice} 

\begin{pozn}
Pokud vztah v bodě \emph{2.} předchozí definice otočíme o 90°, získáme vztah
$$ \left\langle
\minuskriz
\right\rangle = A  \left\langle
\horkriz
\right\rangle+ A^{-1} \left\langle
\vertkriz
\right\rangle. $$
\end{pozn}

\begin{lemma}\label{l01:2}
Pro závorkové polynomy linků, jejichž diagramy obsahují smyčku, platí
\begin{enumerate}
\item
$ \left\langle 
%smycka 
\plussmycka
\right\rangle = -A^{-3} \left\langle 
\odsmycka
  \right\rangle$ 
\item
$ \left\langle 
\minussmycka
  \right\rangle = -A^{3} \left\langle 
\odsmycka
 \right\rangle$
\end{enumerate}
\end{lemma}

\begin{dukaz}
Použitím odvozovacích pravidel dokážeme první bod. 
\begin{equation*}
\left\langle \plussmycka \right\rangle  =  A \left\langle \odsmycka \cup  \pmb{\bigcirc}   \right\rangle + A^{-1} \left\langle \odsmycka
 \right\rangle  \\  = A  (-A^2 - A^{-2} ) \left\langle \odsmycka \right\rangle + A^{-1}  \left\langle \odsmycka  \right\rangle  \\  = -A^{-3} \left\langle \odsmycka \right\rangle
\end{equation*}
Druhý bod se dokáže analogicky.
\end{dukaz}

Dva diagramy znázorňují ekvivalentní linky, pokud lze jeden získat z druhého sérií Reidemeisterových pohybů~\cite{Adams2004}, viz obrázek~\ref{reid}. Z lemmatu ~\ref{l01:2} plyne, že závorkový polynom není invariantní vůči Reidemeisterovu pohybu typu I. Ukážeme, že je invariantní vůči Reidemeisterovým pohybům typu~II a typu~III.

\begin{figure}[h]
\centering 
\begin{subfigure}{1\linewidth}\centering
\begin{tikzpicture}[scale = 0.70]
\clip(-2,-2) rectangle (7,1.3);
\draw (-1,1) .. controls (-2,2) and (-2,-2) .. (-1, -1);
\draw (-1,1) -- (1, -1);
\draw (-1,-1) -- (-0.25, -0.25);
\draw (0.25,0.25) -- (1,1);

\draw [thick, <->] (2,0) -- (3,0);

\draw (5,-1) arc (270:90:1);
\draw (5,-1) -- (7,-1);
\draw (5,1) -- (7,1);

\end{tikzpicture}
\caption{Typ I.} 
\end{subfigure}
\begin{subfigure}{1\linewidth}\centering
\begin{tikzpicture}[scale = 0.70]
\clip(-2,-2) rectangle (7,1.3);

\draw (-1,-1) arc (270:90:1);
%\draw (-1, -1) -- (1, -1);
\draw (-1, -1) -- (-0.25, -1);
\draw (0.25, -1) -- (1, -1);
%\draw (-1, 1) -- (1, 1);
\draw (0, -1.3) -- (0, 1.3);
\draw (4, -1.3) -- (4, 1.3);

\draw (-1, 1) -- (-0.25, 1);
\draw (0.25, 1) -- (1, 1);

\draw [thick, <->] (2,0) -- (3,0);
\draw (5.5,-1) arc (270:90:1);
\draw (5.5,-1) -- (7,-1);
\draw (5.5,1) -- (7,1);

\end{tikzpicture} 
\caption{Typ II.}
\end{subfigure}
\begin{subfigure}{1\linewidth}\centering
\begin{tikzpicture}[scale = 0.70]
\clip(-4.5,-1.65) rectangle (4.5, 1.65);

\draw [thick, <->] (-1/2,0) -- (1/2,0);

\draw (-4,1) -- (-3.58, 0.72);
\draw (-3.4, 0.6) -- (-1,-1);
\draw (-4, -1) -- (-3.58, -0.72);
\draw (-3.4, -0.6) -- (-2.8, -0.2);
\draw (-2.2, 0.2) -- (-1, 1);

\draw (1,-1) -- (2.2, -0.2);
\draw (2.8, 0.2) -- (3.4, 0.6);
\draw (3.58, 0.72) -- (4, 1); 
\draw (1,1) --  (3.4, -0.6) ;
\draw (3.58, -0.72) -- (4, -1);

\draw (-3.5, -1) -- (-3.5, 1);
\draw (3.5,-1) -- (3.5, 1);

\end{tikzpicture}
\caption{Typ III.}
\end{subfigure}
\caption{Reidemeisterovy pohyby.} \label{reid}
\end{figure}

\begin{tvrz}\label{t01:3}
Závorkový polynom je invariantní vůči Reidemeisterovým pohybům typu II a III.
\end{tvrz}
\begin{dukaz}
Použitím odvozovacích pravidel dokážeme invarianci vůči pohybu typu~II.
\begin{equation*}
\begin{split}
\left< \reiddva \right> & = A \left< \FO \right> + A^{-1} \left< \BO \right> \\ & = A \left( A \left< \DO \right> + A^{-1} \left< \EO \right> \right) \\ & + A^{-1} \left( A \left< \CO \right> +  A^{-1} \left< \DO \right> \right) \\ & = \left< \CO \right> + \left( A^2 + A^{-2} \right) \left< \DO \right> + \left( -A^2 + -A^{-2} \right) \left< \DO \right>  \\ & = \left< \CO \right>
\end{split}
\end{equation*}
Invariance vůči pohybu typu III plyne z invariance vůči pohybu typu~II.
\begin{equation*}
\begin{split}
\left< \GO \right> & = A \left< \JO \right>+  A^{-1} \left< \IO \right> \\ & = A \left< \JO \right> + A^{-1} \left< \KO \right>  \\ & = A \left< \JO \right> + A^{-1} \left< \LO \right> = \left< \HO \right>
\end{split}
\end{equation*}
\end{dukaz}

Aby byl závorkový polynom invariantní také vůči Reidemeisterovu pohybu typu~I, je nutné vynásobit jej výrazem, který vyjadřuje míru zakroucení.

\begin{definice}\label{def01:3}
Buď $D$ orientovaný diagram. \emph{Zakroucení} (writhe) $w(D) \in \Z$ je součet znamení orientací všech křížení v $D$.
\end{definice}

\begin{lemma}\label{l01:4}
Zakroucení je invariantní vůči Reidemeisterovým pohybům typu~II a~typu~III.
\end{lemma}
\begin{dukaz}
Uvažujme ta dvě křížení, která jsou v diagramu odstraněna nebo vzniknou pohybem typu~II či pohybem typu~III. Jedno křížení musí vždy být kladné a druhé záporné orientace. Jejich odstranění tedy neovlivní hodnotu zamotání.
\end{dukaz}

\begin{definice}\label{def01:4}
\emph{Normalizovaný závorkový polynom} $X_L(A)$ orientovaného linku $L$ definujeme $$X_L(A) = \left(-A^{-3}\right)^{-w\left(L_+\right)}\langle D \rangle,$$ kde $D$ značí libovolný diagram linku $L$.
\end{definice}

\begin{pozn}
Definice je korektní, neboť následující tvrzení ukazuje, že nezáleží na volbě diagramu.
\end{pozn}

\begin{tvrz}\label{t01:5}
Normalizovaný závorkový polynom je linkový invariant.
\end{tvrz}
\begin{dukaz}
Závorkový polynom i zakroucení jsou podle tvrzení~\ref{t01:3} a~lemmatu~\ref{l01:4} invariantní vůči Reidemeisterovým pohybům typu II a III, invariantní je tedy i jejich součin.

Invariance vůči typu I plyne z lemmatu~\ref{l01:2} a faktu, že křížení~\plussmycka~je vždy kladné a křížení~\minussmycka~vždy záporné.
\end{dukaz}

\begin{tvrz}\label{t01:6}
Při substituci proměnné $A = t^{-1/4}$ je normalizovaný závorkový polynom $X_L(A)$  roven Jonesovu polynomu $V_L(t)$.
\end{tvrz}
\begin{dukaz}
Ověříme, že $X_L\left(t^{-1/4}\right)$ splňuje podmínky v definici~\ref{def01:1}:

\begin{enumerate}
\item
Normalizovaný závorkový polynom je podle předchozího tvrzení linkový invariant.
\item
Pro zamotání triviálního uzlu platí $w\left( \pmb{\bigcirc}\right) = 0$, tedy $$X_{\pmb{\circlearrowleft}} = \left(-A^3\right)^{w( \pmb{\bigcirc})} \left\langle \pmb{\bigcirc} \right\rangle = 1.$$ 
\item
Dokážeme ekvivalentní tvrzení, že $X_L(A)$ splňuje skein vztah~\ref{skein} při substituci $t=A^{-4}$. \\ 
Buď $L_+$, $ L_-$ a $L_0$ diagramy, pak platí $$\left(-A^{-3}\right)^{-w(L_{\pm})} = -A^{\mp 3} \left(-A^{-3}\right)^{-w(L_0)}$$ a dostáváme
\begin{equation*}
\begin{split}
X_{L_+}(A) & = \left(-A^{-3}\right)^{-w(L_+)} \left\langle L_+  \right\rangle = -A^{-3} \left(-A^{-3}\right)^{-w(L_0)}  \left\langle \minuskriz   \right\rangle \\ & = -A^{-3} \left(-A^{-3}\right)^{-w(L_0)} \left(A \left\langle \horkriz  \right\rangle + A^{-1}  \left\langle \vertkriz  \right\rangle\right)
\end{split}
\end{equation*}
\begin{equation*}
\begin{split}
X_{L_-}\left(A\right) & =\left(-A^{-3}\right)^{-w(L_-)} \left\langle L_-  \right\rangle = -A^{3}\left(-A^{-3}\right)^{-w(L_0)} \left\langle \pluskriz   \right\rangle \\ & = -A^{3} \left(-A^{-3}\right)^{-w(L_0)} \left(A \left\langle \vertkriz  \right\rangle + A^{-1}  \left\langle \horkriz  \right\rangle\right) .
\end{split}
\end{equation*}
Levá strana vztahu~\ref{skein} se tedy při substituci $t=A^{-4}$ rovná
\begin{equation*}
\begin{split}
A^4 X_{L_+}\left(t^{1/4}\right)  - A^{-4} X_{L_-} \left(t^{1/4}\right) &  =  \left(-A^3\right)^{-w(L_0)} \left[ -A  \left(A \left\langle \horkriz \right\rangle + A^{-1}  \left\langle \vertkriz \right\rangle \right) \right. \\ & \left. + A^{-1} \left(A \left\langle \vertkriz \right\rangle + A^{-1}  \left\langle \horkriz \right\rangle\right) \right] \\ & = \left(-A^3\right)^{-w(L_0)} \left(A^{-2} -A^2\right) \left\langle \horkriz \right\rangle \\ & =  \left(A^{-2} -A^2\right) X_{L_0},
\end{split}
\end{equation*}
což jsme měli dokázat.
\end{enumerate}
$ $
\end{dukaz}

%%% Druhá kapitola
\chapter{Druhá}


\section{Co v ní}
Je to v tride number P  
Popis algoritmu, vypocet horniho odhadu, dolní odhad pro nějakou třídu uzlů, na které se to rozbije, skripta z počítačové algebry, důkaz správnosti algoritmu
Odhad složitosti?


Bylo by zajímavé identifikovat, pro které typy uzlů je algoritmus efektivní a pro které naopak dosahuje nejhorších výsledků.
Rychle na kanonickych nakreslenich torus uzlu a preclikovych uzlu.
%%% Třetí kapitola

\chapter{Testování algoritmu na datech}

V této sekci představíme výsledky implementace výpočtu Jonesova polynomu. Změříme délku běhu algoritmu a jeho variant na malých uzlech, náhodých uzlech a lincích a na speciálních typech uzlů.

\section{Malé tabulkové uzly}

Prostě že se to dělá a že to je rychlé a že je ta tabulka. Větší smysl to má pro větší.
\section{Náhodné uzly a linky}

\subsection{Generování náhodných linků a uzlů}
Náhodné linky budeme generovat pomocí rovinných grafů, neboť mezi linky a rovinnými grafy existuje vzájemně jednoznačná korespondence, zdroj?. (Jedná se o jinou korespondenci,  než tu mezi linky a rovinnými grafy s vrcholu stupně čtyři popsanou v sekci **).

\subsubsection{Převod rovinného grafu na link}
Rovinný graf z $n$ hranami odpovídá linku s $n$ kříženími. Každé hraně přířadíme náhodně kladné, či záporné znamení a umístíme na ni křížení příslušného typu. Úseky mezi kříženími jsou tím již jednoznačně určené: musí spojovat křížení mezi nejbližšími hranami tak, aby nevznikla žádná další křížení.

OBRÁZEK, KRESLENÝ RUKOU

\subsubsection{Generování náhodných rovinných grafů}
Graf s $n$ hranami získáme následujícím způsobem. Vygenerujeme vhodný počet náhodných bodů v rovině a nalezeneme jejich triangulaci, tj. spojíme body hranami tak, aby byl polygon tvořící konvexní obal bodů rozdělen na trojúhelníky.

\begin{figure}[h]  
\centering 
\begin{tikzpicture}
\draw[fill] (1,1) circle [radius=0.05];
\node [below] at (1,1) {$P_1$};

\draw[fill] (2,-1/2) circle [radius=0.05];
\node [below] at (2,-1/2) {$P_2$};

\draw[fill] (2,3) circle [radius=0.05];
\node [above] at (2,3) {$P_3$};

\draw[fill] (-1,3/2) circle [radius=0.05];
\node [above] at (-1,3/2) {$P_4$};

\draw[fill] (-3,3/2) circle [radius=0.05];
\node [above] at (-3,3/2) {$P_5$};

\draw[fill] (-2,-1) circle [radius=0.05];
\node [below] at (-2,-1) {$P_6$};

\draw (-3,3/2) -- (2,3);
\draw (-3,3/2) -- (-1,3/2);
\draw (-3,3/2) -- (-2,-1);
\draw (-1,3/2)  -- (2,3) ;
\draw (-1,3/2)  -- (-2,-1) ;
\draw (-1,3/2)  -- (1,1) ;
\draw (-1,3/2)  -- (1,1) ;
\draw (2,3)   -- (1,1) ;
\draw (-2,-1) -- (1,1) ;
\draw (-2,-1) -- (2,-1/2);
\draw  (1,1)  -- (2,-1/2);
\draw  (2,3)-- (2,-1/2);
\end{tikzpicture}
\caption{Triangulace šesti bodů}
\end{figure}  

Získali jsme tak rovinný graf s původními body jako vrcholy. Pokud je počet hran menší než $n$, provedeme triangulaci znovu s větším počtem bodů. Pokud je počet hran větší než $n$, odstraníme potřebný počet náhodně zvolených hran.

\subsubsection{Implementace}

Triangulace bodů je snadno implementovatelný geometrický problém. \\ Se získaným rovinným grafem pracujeme jako s množinou vrcholů a k nim příslušným hranám seřazeným v pořadí, jak k danému vrcholu přiléhají. Z této struktury je již možné získat PD notaci příslušného linku.


V PD notaci lze procházkou po vláknu snadno poznat, jestli je vygenerovaný link uzlem. Také jsme v PD notaci jednoduchou operací schopni uzel změnit na alternující, tedy takový, v němž se střídají křížení vedená zespodu a zvrchu.

Jsme tedy schopni nagenerovat uzly, alternující uzly a linky libovolné velikosti.

Takto generované uzly jsou také poměrně \uv{zamotané}, tedy rozmotávací krok algoritmu neodstraní příliš mnoho křížení.

\subsection{Test}
Rozdíl mezi uzly a alternujícími uzly.
Rozdíly algoritmů na uzlech.
Linky?
U uzlů a alt od větších uzlů fitovat data a shrnout.

\section{Torusové uzly}
Rovnou srovnání B a R.

Pak rozebrat ty dva kusy.

Nezapomenou dělat závěry a shrnovat.



\chapter*{Závěr}
\addcontentsline{toc}{chapter}{Závěr}

Nezapomenout dělat závěry průběžně. Shrnovat to. Já vím, co to znamená, ale oni ne! Takže do toho.

Takže hlavně shrnovat ty složitosti. Že to na některých uzlech běží dobře. Nachám to běžet na torus.

Myslet meta - co mi tak ještě chybí, aby to dobře shrnulo Jonesův polynom?
Jakože je to práce o něm? Nebo mám jenom studovat?

%%% Seznam použité literatury
\include{literatura}

%%% Obrázky v bakalářské práci
%%% (pokud jich je malé množství, obvykle není třeba seznam uvádět)
\listoffigures

%%% Tabulky v bakalářské práci (opět nemusí být nutné uvádět)
%%% U matematických prací může být lepší přemístit seznam tabulek na začátek práce.
\listoftables

%%% Použité zkratky v bakalářské práci (opět nemusí být nutné uvádět)
%%% U matematických prací může být lepší přemístit seznam zkratek na začátek práce.

%\chapwithtoc{Seznam použitých zkratek}

%ja
\listofalgorithms
\addcontentsline{toc}{chapter}{Seznam algoritmů}


%%% Přílohy k bakalářské práci, existují-li. Každá příloha musí být alespoň jednou
%%% odkazována z vlastního textu práce. Přílohy se číslují.
%%%
%%% Do tištěné verze se spíše hodí přílohy, které lze číst a prohlížet (dodatečné
%%% tabulky a grafy, různé textové doplňky, ukázky výstupů z počítačových programů,
%%% apod.). Do elektronické verze se hodí přílohy, které budou spíše používány
%%% v elektronické podobě než čteny (zdrojové kódy programů, datové soubory,
%%% interaktivní grafy apod.). Elektronické přílohy se nahrávají do SISu a lze
%%% je také do práce vložit na CD/DVD. Povolené formáty souborů specifikuje
%%% opatření rektora č. 72/2017.
%\appendix
%\chapter{Přílohy}

%\section{První příloha}

\openright
\end{document}
