\documentclass[10pt]{article}

%% Přepneme na českou sazbu a fonty Latin Modern
\usepackage[czech]{babel}
\usepackage{lmodern}
\usepackage[T1]{fontenc}
\usepackage{textcomp}
\usepackage{subcaption}


%% Použité kódování znaků: obvykle latin2, cp1250 nebo utf8:
\usepackage[utf8]{inputenc}
\usepackage{pgf,tikz,pgfplots, subcaption}
\pgfplotsset{compat=1.15}
\usepackage{mathrsfs}
\usetikzlibrary{arrows}
\usetikzlibrary{
  knots,
  hobby,
  decorations.pathreplacing,
  shapes.geometric,
  calc,
  decorations.markings,
  bending
}

\pagestyle{empty}
\begin{document}


\begin{figure}[h]
\centering 
\begin{tikzpicture}[use Hobby shortcut]
\begin{knot}[
  consider self intersections=true,
%  draft mode=crossings,
  ignore endpoint intersections=false,
  flip crossing/.list={6,4,2}
  ]
\node at (0.7,-0.2) {4};
\node at (-0.7,-0.2) {9};
\node at (-0.8,1.3) {3};
\node at (0.8,1.3) {8};
\node at (-2.5,1.3) {7};
\node at (2.5,1.3) {2};
\node at (2.5,-1.7) {5};
\node at (-2.5,-1.7) {10};
\node at (0.7,-1.2) {1};
\node at (-0.7,-1.2) {6};
\strand ([closed]2,2) .. (1.8,0) .. (-2.3,-1) .. (.5,1) .. (-2,2) .. (-1.8,0) .. (2.3,-1) .. (-.5,1) .. (2,2);
\end{knot}
\end{tikzpicture}
\caption{Uzel s PD notací [1, 5, 2, 4], [7, 3, 8, 2], [3, 9, 4, 8], [5, 1, 6, 10], [9,~7,~10,~6].} \label{pd}
\end{figure}



\begin{figure}[h]
\centering 
\begin{subfigure}[t]{0.4\linewidth}\centering
\begin{tikzpicture}
%\clip(-2,-2) rectangle (7,1.3);

\draw (-1,-0.75) arc (270:90:0.75);
\draw (-1, -0.75) -- (1, -0.75);
%\draw (-1, 1) -- (1, 1);
\draw (0, -1.3) -- (0, -1);
\draw (0, -0.6) -- (0, 1.3);

\draw (-1, 0.75) -- (-0.25, 0.75);
\draw (0.25, 0.75) -- (1, 0.75);
\node [below] at (0.3, -0.3) {$a_1$};
\node [below] at (0.3, .7) {$a_2$};
\end{tikzpicture}
\caption{Typ A} 
\end{subfigure}
\begin{subfigure}[t]{0.4\linewidth}\centering
\begin{tikzpicture}
%\clip(-4.5,-1.65) rectangle (4.5, 1.65);


\draw (-4,1) -- (-3.58, 0.72);
\draw (-3.4, 0.6) -- (-1.6,-.6);
\draw (-4, -1) -- (-3.58, -0.72);
\draw (-3.4, -0.6) -- (-2.8, -0.2);
\draw (-2.2, 0.2) -- (-1.6, 0.6);


\draw (-3.5, -1.3) -- (-3.5, 1.3);

\node [below, right] at (-3.3, -0.7) {$b_2$};
\node [above, right] at (-3.3, 0.7) {$b_3$};
\node [right] at (-2.3,0) {$b_1$};
\end{tikzpicture} 
\caption{Typ B}
\end{subfigure}
\begin{subfigure}[t]{0.4\linewidth}\centering
\begin{tikzpicture}
%\clip(-4.5,-1.65) rectangle (4.5, 1.65);


\draw (-4,1) -- (-3.58, 0.72);
\draw (-3.4, 0.6) -- (-1.6,-.6);
\draw (-4, -1) -- (-2.8, -0.2);
\draw (-2.2, 0.2) -- (-1.6, 0.6);


\draw (-3.5, -1.3) -- (-3.5, -0.9);
\draw (-3.5, -0.5) -- (-3.5, 1.3);
\node [below, right] at (-3.3, -0.7) {$c_2$};
\node [above, right] at (-3.3, 0.7) {$c_3$};
\node [right] at (-2.3,0) {$c_1$};
\end{tikzpicture}
\caption{Typ C}
\end{subfigure}
\caption{Typy stěn ohraničených třemi a méně hranami.} \label{stenyobr}
\end{figure}


\end{document}
