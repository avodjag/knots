%%% Třetí kapitola

\chapter{Třetí}

\section{Co v ní}
Experiment, náhodné uzly, experimenty na jinych uzlech, různé algoritmy?
Zatim urcite nechat probehnout na dvanacti uzlech,
Pak na velkych nahodnych uzlech
Pak na nejakych specialnich uzlech podle meho uvazeni
Jak generuji nahodne uzly, alternující uzly, linky,...
Bylo by zajímavé identifikovat, pro které typy uzlů je algoritmus efektivní a pro které naopak dosahuje nejhorších výsledků.

Rychlý na torus: vzít těch 36.
Pokusit se naprogramovat ta spojení koleček.

Nezapomenou dělat závěry.

Co náhodné alternující uzly?

Neměl by být experiment pouze na náhodných alternujících uzlech?

Udělám oboje dohromady!

Takže zvládne můj generátor ještě tohle? Já už nevím, jak to dělal!

Ale řekla bych, že by to mohl zvládnout.
